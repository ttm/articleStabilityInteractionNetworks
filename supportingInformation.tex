% ****** Start of file aipsamp.tex ******
%
%   This file is part of the AIP files in the AIP distribution for REVTeX 4.
%   Version 4.1 of REVTeX, October 2009
%
%   Copyright (c) 2009 American Institute of Physics.

% Use this file as a source of example code for your aip document.
% Use the file aiptemplate.tex as a template for your document.
\documentclass[%
 aip,
 jmp,%
 amsmath,amssymb,
%preprint,%
 reprint,%
%author-year,%
%author-numerical,%
]{revtex4-1}
\usepackage{graphicx}% Include figure files
\usepackage{grffile}
\usepackage{dcolumn}% Align table columns on decimal point
\usepackage{bm}% bold math
%\usepackage[mathlines]{lineno}% Enable numbering of text and display math
%\linenumbers\relax % Commence numbering lines
\usepackage{multirow}
\usepackage{color} % for the notes
\usepackage{etex}
\reserveinserts{58}
%\usepackage{morefloats}
\usepackage{hyperref}
\usepackage{xcolor}
\usepackage{amsmath}
\hypersetup{
        colorlinks,
        linkcolor={red!50!black},
        citecolor={blue!50!black},
        urlcolor={blue!80!black}
}
\usepackage{xr}
\externaldocument{paper}

\maxdeadcycles=1000

\begin{document}

\preprint{XXXXX (preprint)}

%\title[Evolution of interaction networks]{On the evolution of interaction networks: primitive typology of vertex, prominence of measures and activity statistics}% Force line breaks with \\
%\title[Evolution of interaction networks]{On the evolution of interaction networks: a primitive typology of vertex}% Force line breaks with \\
\title[Stability of interaction networks, SUPPORTING INFORMATION]{Stability in human interaction networks: primitive typology of vertex, prominence of measures and time activity statistics, SUPPORTING INFORMATION}% Force line breaks with \\

\author{Renato Fabbri}%
 \homepage{http://ifsc.usp.br/~fabbri/}
 \email{fabbri@usp.br}
  \affiliation{ 
S\~ao Carlos Institute of Physics, University of S\~ao Paulo (IFSC/USP)%\\This line break forced with \textbackslash\textbackslash
}

\author{Vilson V. da Silva Jr.}
  \homepage{http://automata.cc/}
  \email{vilson@void.cc}
  \altaffiliation[Also at ]{IFSC-USP}%Lines break automatically or can be forced with \\

\author{Ricardo Fabbri}
  \homepage{http://www.lems.brown.edu/~rfabbri/}
  \email{rfabbri@iprj.uerj.br}
 \altaffiliation{
Instituto Polit\'ecnico, Universidade Estadual do Rio de Janeiro (IPRJ)
}%Lines break automatically or can be forced with \\

\author{Deborah C. Antunes}
  \homepage{http://lattes.cnpq.br/1065956470701739}
  \email{deborahantunes@gmail.com}
  \altaffiliation{
Curso de Psicologia, Universidade Federal do Cer\'a (UFC)
}%Lines break automatically or can be forced with \\

\author{Marilia M. Pisani}
  \homepage{http://lattes.cnpq.br/6738980149860322}
  \email{marilia.m.pisani@gmail.com}
 \altaffiliation{
Centro de Ciências Naturais e Humanas, Universidade Federal do ABC (CCNH/UFABC)
}%Lines break automatically or can be forced with \\

%
%%\author{Luciano da Fontoura Costa}
%%  \homepage{http://cyvision.ifsc.usp.br/~luciano/}
%%  \email{ldfcosta@gmail.com}
%  \altaffiliation[Also at ]{IFSC-USP}%Lines break automatically or can be forced with \\

%\author{Osvaldo N. Oliveira Jr.}
%  \homepage{www.polimeros.ifsc.usp.br/professors/professor.php?id=4}
%  \email{chu@ifsc.usp.br}
% \altaffiliation[Also at ]{IFSC-USP}%Lines break automatically or can be forced with \\


\date{\today}% It is always \today, today,
             %  but any date may be explicitly specified

\begin{abstract}
 This is the supporting information of the article that reports interaction networks stability by means of three quantitative criteria: activity distribution in time and among participants; a sound classification of vertices in peripheral, intermediary and hub sectors; the combination of basic measures into principal components with greater variance. 
\end{abstract}

\pacs{89.75.Fb,05.65.+b,89.65.-s}% PACS, the Physics and Astronomy
\keywords{complex networks, social network analysis, pattern recognition, statistics, anthropological physics}
\maketitle

These results were produced with the Gmane public domain data and an open python package designed for attaining
these, and related, results. The interested reader should follow Appendix~\ref{scripts} to access both data and rotines.
Inline are results for 4 emails lists: LAD, LAU, MET and CPP, as described in Section~\ref{sec:data}.
Similar results can be reproduced for any number of (Gmane) email lists.
To avoid repeating text of each table for each list, the text is given inline.

\section{Time tables in different scales}\label{sec:time}
Theory presented in Section~\ref{sec:mtime} and results exposed in Section~\ref{constDisc} of the paper~\cite{tpaper}.
\subsection{Circular measures}
The rescaled circular mean $\theta_\mu'$, the standard deviation $S(z)$, the variance $Var(z)$, the circular dispersion $\delta(z)$ and the relation of maximum and minimum incidence at each time unit $\frac{max(incidence}{min(incidence}$. Also, $ \mu_{\frac{max(incidence')}{min(incidence')}} $ and $ \sigma_{\frac{max(incidence')}{min(incidence')} }$ are given for 1000 uniform distribution simulations within the same number of bins and with the same number of samples. Section~\ref{sec:mtime} describes the theoretical background of directional (or circular) statistics.
\begin{table*}[t]
	\caption{LAD circular measures}
\begin{center}
    \begin{tabular}{ |l|| c|c|c|c|c||c|c| }
        \hline
scale & $\theta_\mu'$ & $S(z)$ & $Var(z)$ & $\delta(z)$ & $\frac{max(incidence)}{min(incidence)}$ & $ \mu_{\frac{max(incidence')}{min(incidence')}} $ & $ \sigma_{\frac{max(incidence')}{min(incidence')} } $ \\ \hline\hline
	seconds & --//--  & 3.13  & 0.99  & 9070.17  & 1.28  & 1.29  & 0.05 \\\hline
minutes & --//--  & 3.60  & 1.00  & 205489.40  & 1.22  & 1.29  & 0.05 \\\hline
hours & -9.61  & 1.52  & 0.68  & 4.36  & 9.77  & 1.15  & 0.03 \\\hline
weekdays & -0.03  & 2.03  & 0.87  & 29.28  & 1.72  & 1.05  & 0.02 \\\hline
month days & -0.07  & 2.94  & 0.99  & 2754.16  & 2.21  & 1.17  & 0.03 \\\hline
months & -0.56  & 2.14  & 0.90  & 44.00  & 2.25  & 1.09  & 0.02 \\\hline

    \end{tabular}
\end{center}
\label{tab:circ}
\end{table*}



\subsection{Histograms}

\section{Fraction of participants in each Erd\"os Sector along the timeline}\label{sec:frac}
\subsection{LAD}
\subsection{CPP}

\section{Fraction of participants in each Erd\"os Sector along the timeline}\label{sec:pcat}
\subsection{Betweenness, clustering and degree}
\subsection{Betweenness, clustering, degrees and strengths}
\subsection{Betweenness, clustering, degrees, strengths and symmetry measures}

\nocite{*}
\bibliography{supportingInformation}% Produces the bibliography via BibTeX.
\end{document}
%
% ****** End of file aipsamp.tex ******


