% ****** Start of file aipsamp.tex ******
%
%   This file is part of the AIP files in the AIP distribution for REVTeX 4.
%   Version 4.1 of REVTeX, October 2009
%
%   Copyright (c) 2009 American Institute of Physics.

% Use this file as a source of example code for your aip document.
% Use the file aiptemplate.tex as a template for your document.
\documentclass[%
 aip,
 jmp,%
 amsmath,amssymb,
%preprint,%
 reprint,%
%author-year,%
%author-numerical,%
]{revtex4-1}
\usepackage{graphicx}% Include figure files
\usepackage{grffile}
\usepackage{dcolumn}% Align table columns on decimal point
\usepackage{bm}% bold math
%\usepackage[mathlines]{lineno}% Enable numbering of text and display math
%\linenumbers\relax % Commence numbering lines
\usepackage{multirow}
\usepackage{color} % for the notes
\usepackage{etex}
\reserveinserts{58}
%\usepackage{morefloats}
\usepackage{hyperref}
\usepackage{xcolor}
\usepackage{amsmath}
\hypersetup{
        colorlinks,
        linkcolor={red!50!black},
        citecolor={blue!50!black},
        urlcolor={blue!80!black}
}
\usepackage{placeins}
\usepackage{xr}
\externaldocument{paper}

\maxdeadcycles=1000

\begin{document}

\preprint{XXXXX (preprint)}

%\title[Evolution of interaction networks]{On the evolution of interaction networks: primitive typology of vertex, prominence of measures and activity statistics}% Force line breaks with \\
%\title[Evolution of interaction networks]{On the evolution of interaction networks: a primitive typology of vertex}% Force line breaks with \\
\title[Stability of interaction networks, SUPPORTING INFORMATION]{Stability in human interaction networks: primitive typology of vertex, prominence of measures and time activity statistics, SUPPORTING INFORMATION}% Force line breaks with \\

\author{Renato Fabbri}%
 \homepage{http://ifsc.usp.br/~fabbri/}
 \email{fabbri@usp.br}
  \affiliation{ 
S\~ao Carlos Institute of Physics, University of S\~ao Paulo (IFSC/USP)%\\This line break forced with \textbackslash\textbackslash
}

\author{Vilson V. da Silva Jr.}
  \homepage{http://automata.cc/}
  \email{vilson@void.cc}
  \altaffiliation[Also at ]{IFSC-USP}%Lines break automatically or can be forced with \\

\author{Ricardo Fabbri}
  \homepage{http://www.lems.brown.edu/~rfabbri/}
  \email{rfabbri@iprj.uerj.br}
 \altaffiliation{
Instituto Polit\'ecnico, Universidade Estadual do Rio de Janeiro (IPRJ)
}%Lines break automatically or can be forced with \\

\author{Deborah C. Antunes}
  \homepage{http://lattes.cnpq.br/1065956470701739}
  \email{deborahantunes@gmail.com}
  \altaffiliation{
Curso de Psicologia, Universidade Federal do Cer\'a (UFC)
}%Lines break automatically or can be forced with \\

\author{Marilia M. Pisani}
  \homepage{http://lattes.cnpq.br/6738980149860322}
  \email{marilia.m.pisani@gmail.com}
 \altaffiliation{
Centro de Ciências Naturais e Humanas, Universidade Federal do ABC (CCNH/UFABC)
}%Lines break automatically or can be forced with \\

%
%%\author{Luciano da Fontoura Costa}
%%  \homepage{http://cyvision.ifsc.usp.br/~luciano/}
%%  \email{ldfcosta@gmail.com}
%  \altaffiliation[Also at ]{IFSC-USP}%Lines break automatically or can be forced with \\

%\author{Osvaldo N. Oliveira Jr.}
%  \homepage{www.polimeros.ifsc.usp.br/professors/professor.php?id=4}
%  \email{chu@ifsc.usp.br}
% \altaffiliation[Also at ]{IFSC-USP}%Lines break automatically or can be forced with \\


\date{\today}% It is always \today, today,
             %  but any date may be explicitly specified

\begin{abstract}
 This is the supporting information of the article that reports interaction networks stability by means of three quantitative criteria: activity distribution in time and among participants; a sound classification of vertices in peripheral, intermediary and hub sectors; the combination of basic measures into principal components with greater variance. 
\end{abstract}

\pacs{89.75.Fb,05.65.+b,89.65.-s}% PACS, the Physics and Astronomy
\keywords{complex networks, social network analysis, pattern recognition, statistics, anthropological physics}
\maketitle

These results were produced with the Gmane public domain data and an open python package designed for attaining
these, and related, results. The interested reader should follow Appendix~\ref{scripts} to access both data and rotines.
Inline are results for 4 emails lists: LAD, LAU, MET and CPP, as described in Section~\ref{sec:data}.
Similar results can be reproduced for any number of (Gmane) email lists.
To avoid repeating text of each table for each list, the text is given inline.

\section{Time tables in different scales}\label{sec:time}
Theory presented in Section~\ref{sec:mtime} and results exposed in Section~\ref{constDisc} of the paper~\cite{tpaper}.
\subsection{Time circular measures}
The rescaled circular mean $\theta_\mu'$, the standard deviation $S(z)$, the variance $Var(z)$, the circular dispersion $\delta(z)$ and the relation of maximum and minimum incidence at each time unit $\frac{max(incidence}{min(incidence}$. Also, $ \mu_{\frac{max(incidence')}{min(incidence')}} $ and $ \sigma_{\frac{max(incidence')}{min(incidence')} }$ are given for 1000 uniform distribution simulations within the same number of bins and with the same number of samples. Section~\ref{sec:mtime} describes the theoretical background of directional (or circular) statistics.
\begin{table*}[!h]
	\caption{LAU circular measures}
\begin{center}
    \begin{tabular}{ |l|| c|c|c|c|c||c|c| }
        \hline
scale & $\theta_\mu'$ & $S(z)$ & $Var(z)$ & $\delta(z)$ & $\frac{max(incidence)}{min(incidence)}$ & $ \mu_{\frac{max(incidence')}{min(incidence')}} $ & $ \sigma_{\frac{max(incidence')}{min(incidence')} } $ \\ \hline\hline
	seconds & --//--  & 3.31  & 1.00  & 29337.65  & 1.27  & 1.29  & 0.04 \\\hline
minutes & --//--  & 3.13  & 0.99  & 8879.19  & 1.32  & 1.29  & 0.04 \\\hline
hours & -8.76  & 1.56  & 0.71  & 4.92  & 8.38  & 1.14  & 0.03 \\\hline
weekdays & -0.21  & 2.14  & 0.90  & 45.41  & 1.62  & 1.05  & 0.02 \\\hline
month days & -0.64  & 2.76  & 0.98  & 1001.75  & 1.49  & 1.17  & 0.03 \\\hline
months & 3.55  & 2.30  & 0.93  & 94.53  & 1.57  & 1.09  & 0.02 \\\hline

    \end{tabular}
\end{center}
\label{tab:circ}
\end{table*}
\begin{table*}[!h]
	\caption{LAD circular measures}
\begin{center}
    \begin{tabular}{ |l|| c|c|c|c|c||c|c| }
        \hline
scale & $\theta_\mu'$ & $S(z)$ & $Var(z)$ & $\delta(z)$ & $\frac{max(incidence)}{min(incidence)}$ & $ \mu_{\frac{max(incidence')}{min(incidence')}} $ & $ \sigma_{\frac{max(incidence')}{min(incidence')} } $ \\ \hline\hline
	seconds & --//--  & 3.13  & 0.99  & 9070.17  & 1.28  & 1.29  & 0.05 \\\hline
minutes & --//--  & 3.60  & 1.00  & 205489.40  & 1.22  & 1.29  & 0.05 \\\hline
hours & -9.61  & 1.52  & 0.68  & 4.36  & 9.77  & 1.15  & 0.03 \\\hline
weekdays & -0.03  & 2.03  & 0.87  & 29.28  & 1.72  & 1.05  & 0.02 \\\hline
month days & -0.07  & 2.94  & 0.99  & 2754.16  & 2.21  & 1.17  & 0.03 \\\hline
months & -0.56  & 2.14  & 0.90  & 44.00  & 2.25  & 1.09  & 0.02 \\\hline

    \end{tabular}
\end{center}
\label{tab:circ}
\end{table*}
\begin{table*}[!h]
	\caption{MET circular measures}
\begin{center}
    \begin{tabular}{ |l|| c|c|c|c|c||c|c| }
        \hline
scale & $\theta_\mu'$ & $S(z)$ & $Var(z)$ & $\delta(z)$ & $\frac{max(incidence)}{min(incidence)}$ & $ \mu_{\frac{max(incidence')}{min(incidence')}} $ & $ \sigma_{\frac{max(incidence')}{min(incidence')} } $ \\ \hline\hline
	seconds & --//--  & 3.06  & 0.99  & 5910.47  & 1.27  & 1.29  & 0.04 \\\hline
minutes & --//--  & 3.14  & 0.99  & 9696.29  & 1.34  & 1.29  & 0.04 \\\hline
hours & -9.20  & 1.35  & 0.60  & 2.76  & 19.26  & 1.14  & 0.03 \\\hline
weekdays & -0.27  & 1.86  & 0.82  & 13.82  & 2.89  & 1.05  & 0.02 \\\hline
month days & 3.58  & 2.49  & 0.95  & 237.30  & 1.55  & 1.17  & 0.03 \\\hline
months & -2.92  & 1.73  & 0.78  & 9.20  & 3.04  & 1.09  & 0.02 \\\hline

    \end{tabular}
\end{center}
\label{tab:circ}
\end{table*}
\begin{table*}[!h]
	\caption{CPP circular measures}
\begin{center}
    \begin{tabular}{ |l|| c|c|c|c|c||c|c| }
        \hline
scale & $\theta_\mu'$ & $S(z)$ & $Var(z)$ & $\delta(z)$ & $\frac{max(incidence)}{min(incidence)}$ & $ \mu_{\frac{max(incidence')}{min(incidence')}} $ & $ \sigma_{\frac{max(incidence')}{min(incidence')} } $ \\ \hline\hline
	seconds & --//--  & 3.31  & 1.00  & 28205.46  & 0.79  & 0.78  & 0.03 \\\hline
minutes & --//--  & 3.18  & 0.99  & 12275.59  & 0.79  & 0.78  & 0.03 \\\hline
hours & -9.39  & 1.48  & 0.67  & 3.91  & 0.09  & 0.87  & 0.02 \\\hline
weekdays & -0.17  & 1.83  & 0.81  & 12.66  & 0.39  & 0.95  & 0.01 \\\hline
month days & -10.12  & 3.16  & 0.99  & 10789.17  & 0.65  & 0.85  & 0.02 \\\hline
months & 0.15  & 2.34  & 0.93  & 115.49  & 0.67  & 0.92  & 0.02 \\\hline

    \end{tabular}
\end{center}
\label{tab:circ}
\end{table*}

\FloatBarrier
\subsection{Time histograms}
\subsection{Histograms of activity along the hours of the day}

Activity percentages along the hours of the day. Higher activity was observed between noon and 6pm, followed by the time period between 6pm and midnight. Around 2/3 of the whole activity takes place from noon to midnight. Nevertheless, the activity peak occurs around midday, with a slight skew toward one hour before noon.
\begin{table}[!h]
	\caption{LAU activity along the hours of the day}
	\footnotesize
	\begin{center}
\begin{tabular}{l || c | c | c | c | c | c |}\hline
 & 1h & 2h & 3h & 4h & 6h & 12h \\\hline
0h & \multirow{1}{*}{ 3.58 }  & \multirow{2}{*}{ 5.80 }  & \multirow{3}{*}{ 7.43 }  & \multirow{4}{*}{ 8.49 }  & \multirow{6}{*}{ 10.14 }  & \multirow{12}{*}{ 36.88 }  \\\cline{1-1}
1h & \multirow{1}{*}{ 2.22 }  & & & & & \\\cline{1-1}\cline{2-2}
2h & \multirow{1}{*}{ 1.63 }  & \multirow{2}{*}{ 2.69 }  & & & & \\\cline{1-1}\cline{3-3}
3h & \multirow{1}{*}{ 1.06 }  & & \multirow{3}{*}{ 2.72 }  & & & \\\cline{1-1}\cline{2-2}\cline{4-4}
4h & \multirow{1}{*}{ 0.84 }  & \multirow{2}{*}{ 1.66 }  & & \multirow{4}{*}{ 5.20 }  & & \\\cline{1-1}
5h & \multirow{1}{*}{ 0.82 }  & & & & & \\\cline{1-1}\cline{2-2}\cline{3-3}\cline{5-5}
6h & \multirow{1}{*}{ 1.17 }  & \multirow{2}{*}{ 3.54 }  & \multirow{3}{*}{ 7.07 }  & & \multirow{6}{*}{ 26.74 }  & \\\cline{1-1}
7h & \multirow{1}{*}{ 2.37 }  & & & & & \\\cline{1-1}\cline{2-2}\cline{4-4}
8h & \multirow{1}{*}{ 3.53 }  & \multirow{2}{*}{ 9.57 }  & & \multirow{4}{*}{ 23.20 }  & & \\\cline{1-1}\cline{3-3}
9h & \multirow{1}{*}{ 6.04 }  & & \multirow{3}{*}{ 19.67 }  & & & \\\cline{1-1}\cline{2-2}
10h & \multirow{1}{*}{ 6.83 }  & \multirow{2}{*}{ 13.62 }  & & & & \\\cline{1-1}
11h & \multirow{1}{*}{ 6.79 }  & & & & & \\\cline{1-1}\cline{2-2}\cline{3-3}\cline{4-4}\cline{5-5}\cline{6-6}
12h & \multirow{1}{*}{ 6.11 }  & \multirow{2}{*}{ 12.36 }  & \multirow{3}{*}{ 18.75 }  & \multirow{4}{*}{ 24.68 }  & \multirow{6}{*}{ 35.66 }  & \multirow{12}{*}{ 63.12 }  \\\cline{1-1}
13h & \multirow{1}{*}{ 6.26 }  & & & & & \\\cline{1-1}\cline{2-2}
14h & \multirow{1}{*}{ 6.38 }  & \multirow{2}{*}{ 12.31 }  & & & & \\\cline{1-1}\cline{3-3}
15h & \multirow{1}{*}{ 5.93 }  & & \multirow{3}{*}{ 16.91 }  & & & \\\cline{1-1}\cline{2-2}\cline{4-4}
16h & \multirow{1}{*}{ 5.52 }  & \multirow{2}{*}{ 10.98 }  & & \multirow{4}{*}{ 20.73 }  & & \\\cline{1-1}
17h & \multirow{1}{*}{ 5.46 }  & & & & & \\\cline{1-1}\cline{2-2}\cline{3-3}\cline{5-5}
18h & \multirow{1}{*}{ 5.23 }  & \multirow{2}{*}{ 9.75 }  & \multirow{3}{*}{ 14.30 }  & & \multirow{6}{*}{ 27.46 }  & \\\cline{1-1}
19h & \multirow{1}{*}{ 4.52 }  & & & & & \\\cline{1-1}\cline{2-2}\cline{4-4}
20h & \multirow{1}{*}{ 4.55 }  & \multirow{2}{*}{ 8.97 }  & & \multirow{4}{*}{ 17.71 }  & & \\\cline{1-1}\cline{3-3}
21h & \multirow{1}{*}{ 4.42 }  & & \multirow{3}{*}{ 13.16 }  & & & \\\cline{1-1}\cline{2-2}
22h & \multirow{1}{*}{ 4.51 }  & \multirow{2}{*}{ 8.74 }  & & & & \\\cline{1-1}
23h & \multirow{1}{*}{ 4.23 }  & & & & & \\\cline{1-1}\cline{2-2}\cline{3-3}\cline{4-4}\cline{5-5}\cline{6-6}
\end{tabular}
\end{center}
\end{table}

\begin{table}[!h]
	\caption{LAD activity along the hours of the day}
	\footnotesize
	\begin{center}
\begin{tabular}{l || c | c | c | c | c | c |}\hline
 & 1h & 2h & 3h & 4h & 6h & 12h \\\hline
0h & \multirow{1}{*}{ 4.01 }  & \multirow{2}{*}{ 6.53 }  & \multirow{3}{*}{ 8.32 }  & \multirow{4}{*}{ 9.37 }  & \multirow{6}{*}{ 10.78 }  & \multirow{12}{*}{ 33.11 }  \\\cline{1-1}
1h & \multirow{1}{*}{ 2.52 }  & & & & & \\\cline{1-1}\cline{2-2}
2h & \multirow{1}{*}{ 1.79 }  & \multirow{2}{*}{ 2.84 }  & & & & \\\cline{1-1}\cline{3-3}
3h & \multirow{1}{*}{ 1.06 }  & & \multirow{3}{*}{ 2.46 }  & & & \\\cline{1-1}\cline{2-2}\cline{4-4}
4h & \multirow{1}{*}{ 0.75 }  & \multirow{2}{*}{ 1.40 }  & & \multirow{4}{*}{ 3.81 }  & & \\\cline{1-1}
5h & \multirow{1}{*}{ 0.66 }  & & & & & \\\cline{1-1}\cline{2-2}\cline{3-3}\cline{5-5}
6h & \multirow{1}{*}{ 0.85 }  & \multirow{2}{*}{ 2.41 }  & \multirow{3}{*}{ 5.36 }  & & \multirow{6}{*}{ 22.33 }  & \\\cline{1-1}
7h & \multirow{1}{*}{ 1.56 }  & & & & & \\\cline{1-1}\cline{2-2}\cline{4-4}
8h & \multirow{1}{*}{ 2.95 }  & \multirow{2}{*}{ 7.61 }  & & \multirow{4}{*}{ 19.93 }  & & \\\cline{1-1}\cline{3-3}
9h & \multirow{1}{*}{ 4.66 }  & & \multirow{3}{*}{ 16.98 }  & & & \\\cline{1-1}\cline{2-2}
10h & \multirow{1}{*}{ 5.92 }  & \multirow{2}{*}{ 12.32 }  & & & & \\\cline{1-1}
11h & \multirow{1}{*}{ 6.40 }  & & & & & \\\cline{1-1}\cline{2-2}\cline{3-3}\cline{4-4}\cline{5-5}\cline{6-6}
12h & \multirow{1}{*}{ 6.41 }  & \multirow{2}{*}{ 12.53 }  & \multirow{3}{*}{ 18.85 }  & \multirow{4}{*}{ 24.82 }  & \multirow{6}{*}{ 37.24 }  & \multirow{12}{*}{ 66.89 }  \\\cline{1-1}
13h & \multirow{1}{*}{ 6.12 }  & & & & & \\\cline{1-1}\cline{2-2}
14h & \multirow{1}{*}{ 6.32 }  & \multirow{2}{*}{ 12.29 }  & & & & \\\cline{1-1}\cline{3-3}
15h & \multirow{1}{*}{ 5.97 }  & & \multirow{3}{*}{ 18.39 }  & & & \\\cline{1-1}\cline{2-2}\cline{4-4}
16h & \multirow{1}{*}{ 6.40 }  & \multirow{2}{*}{ 12.42 }  & & \multirow{4}{*}{ 23.44 }  & & \\\cline{1-1}
17h & \multirow{1}{*}{ 6.02 }  & & & & & \\\cline{1-1}\cline{2-2}\cline{3-3}\cline{5-5}
18h & \multirow{1}{*}{ 5.99 }  & \multirow{2}{*}{ 11.02 }  & \multirow{3}{*}{ 15.65 }  & & \multirow{6}{*}{ 29.65 }  & \\\cline{1-1}
19h & \multirow{1}{*}{ 5.03 }  & & & & & \\\cline{1-1}\cline{2-2}\cline{4-4}
20h & \multirow{1}{*}{ 4.63 }  & \multirow{2}{*}{ 9.22 }  & & \multirow{4}{*}{ 18.63 }  & & \\\cline{1-1}\cline{3-3}
21h & \multirow{1}{*}{ 4.59 }  & & \multirow{3}{*}{ 14.00 }  & & & \\\cline{1-1}\cline{2-2}
22h & \multirow{1}{*}{ 4.88 }  & \multirow{2}{*}{ 9.41 }  & & & & \\\cline{1-1}
23h & \multirow{1}{*}{ 4.53 }  & & & & & \\\cline{1-1}\cline{2-2}\cline{3-3}\cline{4-4}\cline{5-5}\cline{6-6}
\end{tabular}
\end{center}
\end{table}

\begin{table}[!h]
	\caption{MET activity along the hours of the day}
	\footnotesize
	\begin{center}
\begin{tabular}{| l || c | c | c | c | c | c |}\hline
 & 1h & 2h & 3h & 4h & 6h & 12h \\\hline
0h & \multirow{1}{*}{ 2.87 }  & \multirow{2}{*}{ 4.64 }  & \multirow{3}{*}{ 5.67 }  & \multirow{4}{*}{ 6.31 }  & \multirow{6}{*}{ 7.15 }  & \multirow{12}{*}{ 29.33 }  \\\cline{2-2}
1h & \multirow{1}{*}{ 1.77 }  & & & & & \\\cline{2-2}\cline{3-3}
2h & \multirow{1}{*}{ 1.04 }  & \multirow{2}{*}{ 1.67 }  & & & & \\\cline{2-2}\cline{4-4}
3h & \multirow{1}{*}{ 0.64 }  & & \multirow{3}{*}{ 1.48 }  & & & \\\cline{2-2}\cline{3-3}\cline{5-5}
4h & \multirow{1}{*}{ 0.47 }  & \multirow{2}{*}{ 0.85 }  & & \multirow{4}{*}{ 2.89 }  & & \\\cline{2-2}
5h & \multirow{1}{*}{ 0.38 }  & & & & & \\\cline{2-2}\cline{3-3}\cline{4-4}\cline{6-6}
6h & \multirow{1}{*}{ 0.72 }  & \multirow{2}{*}{ 2.04 }  & \multirow{3}{*}{ 4.71 }  & & \multirow{6}{*}{ 22.18 }  & \\\cline{2-2}
7h & \multirow{1}{*}{ 1.33 }  & & & & & \\\cline{2-2}\cline{3-3}\cline{5-5}
8h & \multirow{1}{*}{ 2.67 }  & \multirow{2}{*}{ 7.07 }  & & \multirow{4}{*}{ 20.14 }  & & \\\cline{2-2}\cline{4-4}
9h & \multirow{1}{*}{ 4.40 }  & & \multirow{3}{*}{ 17.47 }  & & & \\\cline{2-2}\cline{3-3}
10h & \multirow{1}{*}{ 6.29 }  & \multirow{2}{*}{ 13.07 }  & & & & \\\cline{2-2}
11h & \multirow{1}{*}{ 6.78 }  & & & & & \\\cline{2-2}\cline{3-3}\cline{4-4}\cline{5-5}\cline{6-6}\cline{7-7}
12h & \multirow{1}{*}{ 7.33 }  & \multirow{2}{*}{ 14.41 }  & \multirow{3}{*}{ 21.50 }  & \multirow{4}{*}{ 28.65 }  & \multirow{6}{*}{ 42.22 }  & \multirow{12}{*}{ 70.67 }  \\\cline{2-2}
13h & \multirow{1}{*}{ 7.08 }  & & & & & \\\cline{2-2}\cline{3-3}
14h & \multirow{1}{*}{ 7.09 }  & \multirow{2}{*}{ 14.24 }  & & & & \\\cline{2-2}\cline{4-4}
15h & \multirow{1}{*}{ 7.14 }  & & \multirow{3}{*}{ 20.72 }  & & & \\\cline{2-2}\cline{3-3}\cline{5-5}
16h & \multirow{1}{*}{ 6.68 }  & \multirow{2}{*}{ 13.58 }  & & \multirow{4}{*}{ 24.79 }  & & \\\cline{2-2}
17h & \multirow{1}{*}{ 6.89 }  & & & & & \\\cline{2-2}\cline{3-3}\cline{4-4}\cline{6-6}
18h & \multirow{1}{*}{ 5.99 }  & \multirow{2}{*}{ 11.22 }  & \multirow{3}{*}{ 16.19 }  & & \multirow{6}{*}{ 28.44 }  & \\\cline{2-2}
19h & \multirow{1}{*}{ 5.23 }  & & & & & \\\cline{2-2}\cline{3-3}\cline{5-5}
20h & \multirow{1}{*}{ 4.98 }  & \multirow{2}{*}{ 9.34 }  & & \multirow{4}{*}{ 17.22 }  & & \\\cline{2-2}\cline{4-4}
21h & \multirow{1}{*}{ 4.37 }  & & \multirow{3}{*}{ 12.25 }  & & & \\\cline{2-2}\cline{3-3}
22h & \multirow{1}{*}{ 4.24 }  & \multirow{2}{*}{ 7.88 }  & & & & \\\cline{2-2}
23h & \multirow{1}{*}{ 3.64 }  & & & & & \\\cline{2-2}\cline{3-3}\cline{4-4}\cline{5-5}\cline{6-6}\cline{7-7}
\hline\end{tabular}
\end{center}
\end{table}

\begin{table}[!h]
	\caption{CPP activity along the hours of the day}
	\footnotesize
	\begin{center}
\begin{tabular}{l || c | c | c | c | c | c |}\hline
 & 1h & 2h & 3h & 4h & 6h & 12h \\\hline
0h & \multirow{1}{*}{ 3.66 }  & \multirow{2}{*}{ 6.42 }  & \multirow{3}{*}{ 8.20 }  & \multirow{4}{*}{ 9.30 }  & \multirow{6}{*}{ 10.67 }  & \multirow{12}{*}{ 33.76 }  \\\cline{1-1}
1h & \multirow{1}{*}{ 2.76 }  & & & & & \\\cline{1-1}\cline{2-2}
2h & \multirow{1}{*}{ 1.79 }  & \multirow{2}{*}{ 2.88 }  & & & & \\\cline{1-1}\cline{3-3}
3h & \multirow{1}{*}{ 1.10 }  & & \multirow{3}{*}{ 2.47 }  & & & \\\cline{1-1}\cline{2-2}\cline{4-4}
4h & \multirow{1}{*}{ 0.68 }  & \multirow{2}{*}{ 1.37 }  & & \multirow{4}{*}{ 3.44 }  & & \\\cline{1-1}
5h & \multirow{1}{*}{ 0.69 }  & & & & & \\\cline{1-1}\cline{2-2}\cline{3-3}\cline{5-5}
6h & \multirow{1}{*}{ 0.83 }  & \multirow{2}{*}{ 2.07 }  & \multirow{3}{*}{ 4.35 }  & & \multirow{6}{*}{ 23.09 }  & \\\cline{1-1}
7h & \multirow{1}{*}{ 1.24 }  & & & & & \\\cline{1-1}\cline{2-2}\cline{4-4}
8h & \multirow{1}{*}{ 2.28 }  & \multirow{2}{*}{ 6.80 }  & & \multirow{4}{*}{ 21.03 }  & & \\\cline{1-1}\cline{3-3}
9h & \multirow{1}{*}{ 4.52 }  & & \multirow{3}{*}{ 18.75 }  & & & \\\cline{1-1}\cline{2-2}
10h & \multirow{1}{*}{ 6.62 }  & \multirow{2}{*}{ 14.23 }  & & & & \\\cline{1-1}
11h & \multirow{1}{*}{ 7.61 }  & & & & & \\\cline{1-1}\cline{2-2}\cline{3-3}\cline{4-4}\cline{5-5}\cline{6-6}
12h & \multirow{1}{*}{ 6.44 }  & \multirow{2}{*}{ 12.48 }  & \multirow{3}{*}{ 18.95 }  & \multirow{4}{*}{ 25.05 }  & \multirow{6}{*}{ 37.63 }  & \multirow{12}{*}{ 66.24 }  \\\cline{1-1}
13h & \multirow{1}{*}{ 6.04 }  & & & & & \\\cline{1-1}\cline{2-2}
14h & \multirow{1}{*}{ 6.47 }  & \multirow{2}{*}{ 12.57 }  & & & & \\\cline{1-1}\cline{3-3}
15h & \multirow{1}{*}{ 6.10 }  & & \multirow{3}{*}{ 18.68 }  & & & \\\cline{1-1}\cline{2-2}\cline{4-4}
16h & \multirow{1}{*}{ 6.22 }  & \multirow{2}{*}{ 12.58 }  & & \multirow{4}{*}{ 23.60 }  & & \\\cline{1-1}
17h & \multirow{1}{*}{ 6.36 }  & & & & & \\\cline{1-1}\cline{2-2}\cline{3-3}\cline{5-5}
18h & \multirow{1}{*}{ 6.01 }  & \multirow{2}{*}{ 11.02 }  & \multirow{3}{*}{ 15.88 }  & & \multirow{6}{*}{ 28.61 }  & \\\cline{1-1}
19h & \multirow{1}{*}{ 5.02 }  & & & & & \\\cline{1-1}\cline{2-2}\cline{4-4}
20h & \multirow{1}{*}{ 4.85 }  & \multirow{2}{*}{ 9.23 }  & & \multirow{4}{*}{ 17.59 }  & & \\\cline{1-1}\cline{3-3}
21h & \multirow{1}{*}{ 4.38 }  & & \multirow{3}{*}{ 12.73 }  & & & \\\cline{1-1}\cline{2-2}
22h & \multirow{1}{*}{ 4.06 }  & \multirow{2}{*}{ 8.36 }  & & & & \\\cline{1-1}
23h & \multirow{1}{*}{ 4.30 }  & & & & & \\\cline{1-1}\cline{2-2}\cline{3-3}\cline{4-4}\cline{5-5}\cline{6-6}
\end{tabular}
\end{center}
\end{table}

\FloatBarrier
\subsection{Histograms of activity along the days of the week}

Activity percentages along the days of the week. Higher activity was observed during weekdays, with a decrease of activity on weekends of at least one third and two thirds in extreme cases.

\begin{table}[!h]
\begin{center}
    \begin{tabular}{ | l |  c | c | c | c | c |   c | c |}
        \hline
        & Mon & Tue & Wed & Thu & Fri & Sat & Sun  \\ \hline
	LAU & 15.71  & 15.81  & 15.88  & 16.43  & 15.14  & {\bf 10.13}  & {\bf 10.91} \\\hline
LAD & 14.92  & 17.75  & 17.01  & 15.41  & 14.21  & {\bf 10.40}  & {\bf 10.31} \\\hline
MET & 17.53  & 17.54  & 16.43  & 17.06  & 17.46  & {\bf 7.92 }  & {\bf 6.06 }\\\hline
CPP & 17.06  & 17.43  & 17.61  & 17.13  & 16.30  & {\bf 6.81 }  & {\bf 7.67 }\\\hline

    \end{tabular}
\end{center}
\label{tab:win}
\end{table}

\FloatBarrier
\subsection{Histograms of activity along the days of the month}
	No significant variation of activity in the days along the month was observed. One cannot point much more than a - probably not statistically relevant - tendency of first and second weeks to be more active. The most important trait seems to be homogeneity.
\begin{table}[!h]
	\caption{LAU activity along the days of the month.}
	\footnotesize
	\begin{center}
\begin{tabular}{| l || c | c | c | c |}\hline
 & 1 day & 5 & 10 & 15 days \\\hline
1 & \multirow{1}{*}{ 3.36 }  & \multirow{5}{*}{ 16.21 }  & \multirow{10}{*}{ 33.71 }  & \multirow{15}{*}{ 50.82 }  \\\cline{2-2}
2 & \multirow{1}{*}{ 3.43 }  & & & \\\cline{2-2}
3 & \multirow{1}{*}{ 3.31 }  & & & \\\cline{2-2}
4 & \multirow{1}{*}{ 3.37 }  & & & \\\cline{2-2}
5 & \multirow{1}{*}{ 2.75 }  & & & \\\cline{2-2}\cline{3-3}
6 & \multirow{1}{*}{ 3.03 }  & \multirow{5}{*}{ 17.50 }  & & \\\cline{2-2}
7 & \multirow{1}{*}{ 3.93 }  & & & \\\cline{2-2}
8 & \multirow{1}{*}{ 3.62 }  & & & \\\cline{2-2}
9 & \multirow{1}{*}{ 3.84 }  & & & \\\cline{2-2}
10 & \multirow{1}{*}{ 3.09 }  & & & \\\cline{2-2}\cline{3-3}\cline{4-4}
11 & \multirow{1}{*}{ 3.20 }  & \multirow{5}{*}{ 17.11 }  & \multirow{10}{*}{ 34.02 }  & \\\cline{2-2}
12 & \multirow{1}{*}{ 3.40 }  & & & \\\cline{2-2}
13 & \multirow{1}{*}{ 3.67 }  & & & \\\cline{2-2}
14 & \multirow{1}{*}{ 3.71 }  & & & \\\cline{2-2}
15 & \multirow{1}{*}{ 3.14 }  & & & \\\cline{2-2}\cline{3-3}\cline{5-5}
16 & \multirow{1}{*}{ 3.08 }  & \multirow{5}{*}{ 16.91 }  & & \multirow{15}{*}{ 49.18 }  \\\cline{2-2}
17 & \multirow{1}{*}{ 3.13 }  & & & \\\cline{2-2}
18 & \multirow{1}{*}{ 3.43 }  & & & \\\cline{2-2}
19 & \multirow{1}{*}{ 3.61 }  & & & \\\cline{2-2}
20 & \multirow{1}{*}{ 3.67 }  & & & \\\cline{2-2}\cline{3-3}\cline{4-4}
21 & \multirow{1}{*}{ 3.60 }  & \multirow{5}{*}{ 15.43 }  & \multirow{10}{*}{ 32.27 }  & \\\cline{2-2}
22 & \multirow{1}{*}{ 3.42 }  & & & \\\cline{2-2}
23 & \multirow{1}{*}{ 2.80 }  & & & \\\cline{2-2}
24 & \multirow{1}{*}{ 2.64 }  & & & \\\cline{2-2}
25 & \multirow{1}{*}{ 2.97 }  & & & \\\cline{2-2}\cline{3-3}
26 & \multirow{1}{*}{ 3.06 }  & \multirow{5}{*}{ 16.85 }  & & \\\cline{2-2}
27 & \multirow{1}{*}{ 2.69 }  & & & \\\cline{2-2}
28 & \multirow{1}{*}{ 3.79 }  & & & \\\cline{2-2}
29 & \multirow{1}{*}{ 3.75 }  & & & \\\cline{2-2}
30 & \multirow{1}{*}{ 3.57 }  & & & \\\cline{2-2}\cline{3-3}\cline{4-4}\cline{5-5}
\hline\end{tabular}
\end{center}
\label{tab:min}
\end{table}
\begin{table}[!h]
	\caption{LAD activity along the days of the month.}
	\footnotesize
	\begin{center}
\begin{tabular}{| l || c | c | c | c |}\hline
 & 1 day & 5 & 10 & 15 days \\\hline
1 & \multirow{1}{*}{ 5.39 }  & \multirow{5}{*}{ 17.53 }  & \multirow{10}{*}{ 35.28 }  & \multirow{15}{*}{ 52.09 }  \\\cline{2-2}
2 & \multirow{1}{*}{ 2.87 }  & & & \\\cline{2-2}
3 & \multirow{1}{*}{ 2.89 }  & & & \\\cline{2-2}
4 & \multirow{1}{*}{ 3.53 }  & & & \\\cline{2-2}
5 & \multirow{1}{*}{ 2.85 }  & & & \\\cline{2-2}\cline{3-3}
6 & \multirow{1}{*}{ 3.36 }  & \multirow{5}{*}{ 17.75 }  & & \\\cline{2-2}
7 & \multirow{1}{*}{ 3.03 }  & & & \\\cline{2-2}
8 & \multirow{1}{*}{ 3.77 }  & & & \\\cline{2-2}
9 & \multirow{1}{*}{ 4.04 }  & & & \\\cline{2-2}
10 & \multirow{1}{*}{ 3.55 }  & & & \\\cline{2-2}\cline{3-3}\cline{4-4}
11 & \multirow{1}{*}{ 3.00 }  & \multirow{5}{*}{ 16.81 }  & \multirow{10}{*}{ 32.71 }  & \\\cline{2-2}
12 & \multirow{1}{*}{ 3.13 }  & & & \\\cline{2-2}
13 & \multirow{1}{*}{ 3.81 }  & & & \\\cline{2-2}
14 & \multirow{1}{*}{ 3.46 }  & & & \\\cline{2-2}
15 & \multirow{1}{*}{ 3.40 }  & & & \\\cline{2-2}\cline{3-3}\cline{5-5}
16 & \multirow{1}{*}{ 2.84 }  & \multirow{5}{*}{ 15.90 }  & & \multirow{15}{*}{ 47.91 }  \\\cline{2-2}
17 & \multirow{1}{*}{ 2.95 }  & & & \\\cline{2-2}
18 & \multirow{1}{*}{ 3.51 }  & & & \\\cline{2-2}
19 & \multirow{1}{*}{ 3.42 }  & & & \\\cline{2-2}
20 & \multirow{1}{*}{ 3.17 }  & & & \\\cline{2-2}\cline{3-3}\cline{4-4}
21 & \multirow{1}{*}{ 2.49 }  & \multirow{5}{*}{ 15.21 }  & \multirow{10}{*}{ 32.01 }  & \\\cline{2-2}
22 & \multirow{1}{*}{ 3.51 }  & & & \\\cline{2-2}
23 & \multirow{1}{*}{ 3.18 }  & & & \\\cline{2-2}
24 & \multirow{1}{*}{ 2.97 }  & & & \\\cline{2-2}
25 & \multirow{1}{*}{ 3.06 }  & & & \\\cline{2-2}\cline{3-3}
26 & \multirow{1}{*}{ 3.84 }  & \multirow{5}{*}{ 16.80 }  & & \\\cline{2-2}
27 & \multirow{1}{*}{ 3.85 }  & & & \\\cline{2-2}
28 & \multirow{1}{*}{ 3.37 }  & & & \\\cline{2-2}
29 & \multirow{1}{*}{ 3.30 }  & & & \\\cline{2-2}
30 & \multirow{1}{*}{ 2.44 }  & & & \\\cline{2-2}\cline{3-3}\cline{4-4}\cline{5-5}
\hline\end{tabular}
\end{center}
\label{tab:min}
\end{table}
\begin{table}[!h]
	\caption{MET activity along the days of the month.}
	\footnotesize
	\begin{center}
\begin{tabular}{| l || c | c | c | c |}\hline
 & 1 day & 5 & 10 & 15 days \\\hline
1 & \multirow{1}{*}{ 3.05 }  & \multirow{5}{*}{ 18.25 }  & \multirow{10}{*}{ 35.24 }  & \multirow{15}{*}{ 50.96 }  \\\cline{2-2}
2 & \multirow{1}{*}{ 3.38 }  & & & \\\cline{2-2}
3 & \multirow{1}{*}{ 3.62 }  & & & \\\cline{2-2}
4 & \multirow{1}{*}{ 4.25 }  & & & \\\cline{2-2}
5 & \multirow{1}{*}{ 3.94 }  & & & \\\cline{2-2}\cline{3-3}
6 & \multirow{1}{*}{ 3.73 }  & \multirow{5}{*}{ 16.98 }  & & \\\cline{2-2}
7 & \multirow{1}{*}{ 3.17 }  & & & \\\cline{2-2}
8 & \multirow{1}{*}{ 3.26 }  & & & \\\cline{2-2}
9 & \multirow{1}{*}{ 3.56 }  & & & \\\cline{2-2}
10 & \multirow{1}{*}{ 3.26 }  & & & \\\cline{2-2}\cline{3-3}\cline{4-4}
11 & \multirow{1}{*}{ 3.81 }  & \multirow{5}{*}{ 15.73 }  & \multirow{10}{*}{ 31.98 }  & \\\cline{2-2}
12 & \multirow{1}{*}{ 2.91 }  & & & \\\cline{2-2}
13 & \multirow{1}{*}{ 3.30 }  & & & \\\cline{2-2}
14 & \multirow{1}{*}{ 2.75 }  & & & \\\cline{2-2}
15 & \multirow{1}{*}{ 2.95 }  & & & \\\cline{2-2}\cline{3-3}\cline{5-5}
16 & \multirow{1}{*}{ 3.36 }  & \multirow{5}{*}{ 16.25 }  & & \multirow{15}{*}{ 49.04 }  \\\cline{2-2}
17 & \multirow{1}{*}{ 3.16 }  & & & \\\cline{2-2}
18 & \multirow{1}{*}{ 3.44 }  & & & \\\cline{2-2}
19 & \multirow{1}{*}{ 3.36 }  & & & \\\cline{2-2}
20 & \multirow{1}{*}{ 2.93 }  & & & \\\cline{2-2}\cline{3-3}\cline{4-4}
21 & \multirow{1}{*}{ 3.20 }  & \multirow{5}{*}{ 15.79 }  & \multirow{10}{*}{ 32.78 }  & \\\cline{2-2}
22 & \multirow{1}{*}{ 3.11 }  & & & \\\cline{2-2}
23 & \multirow{1}{*}{ 3.60 }  & & & \\\cline{2-2}
24 & \multirow{1}{*}{ 2.74 }  & & & \\\cline{2-2}
25 & \multirow{1}{*}{ 3.13 }  & & & \\\cline{2-2}\cline{3-3}
26 & \multirow{1}{*}{ 3.13 }  & \multirow{5}{*}{ 16.99 }  & & \\\cline{2-2}
27 & \multirow{1}{*}{ 3.07 }  & & & \\\cline{2-2}
28 & \multirow{1}{*}{ 3.61 }  & & & \\\cline{2-2}
29 & \multirow{1}{*}{ 3.60 }  & & & \\\cline{2-2}
30 & \multirow{1}{*}{ 3.57 }  & & & \\\cline{2-2}\cline{3-3}\cline{4-4}\cline{5-5}
\hline\end{tabular}
\end{center}
\label{tab:min}
\end{table}
\begin{table}[!h]
	\caption{CPP activity along the days of the month.}
	\footnotesize
	\begin{center}
\begin{tabular}{| l || c | c | c | c |}\hline
 & 1 day & 5 & 10 & 15 days \\\hline
1 & \multirow{1}{*}{ 5.00 }  & \multirow{5}{*}{ 18.21 }  & \multirow{10}{*}{ 33.84 }  & \multirow{15}{*}{ 51.29 }  \\\cline{2-2}
2 & \multirow{1}{*}{ 3.02 }  & & & \\\cline{2-2}
3 & \multirow{1}{*}{ 3.65 }  & & & \\\cline{2-2}
4 & \multirow{1}{*}{ 3.18 }  & & & \\\cline{2-2}
5 & \multirow{1}{*}{ 3.37 }  & & & \\\cline{2-2}\cline{3-3}
6 & \multirow{1}{*}{ 3.22 }  & \multirow{5}{*}{ 15.63 }  & & \\\cline{2-2}
7 & \multirow{1}{*}{ 3.54 }  & & & \\\cline{2-2}
8 & \multirow{1}{*}{ 2.59 }  & & & \\\cline{2-2}
9 & \multirow{1}{*}{ 2.87 }  & & & \\\cline{2-2}
10 & \multirow{1}{*}{ 3.41 }  & & & \\\cline{2-2}\cline{3-3}\cline{4-4}
11 & \multirow{1}{*}{ 3.42 }  & \multirow{5}{*}{ 17.46 }  & \multirow{10}{*}{ 33.00 }  & \\\cline{2-2}
12 & \multirow{1}{*}{ 3.50 }  & & & \\\cline{2-2}
13 & \multirow{1}{*}{ 3.30 }  & & & \\\cline{2-2}
14 & \multirow{1}{*}{ 3.48 }  & & & \\\cline{2-2}
15 & \multirow{1}{*}{ 3.76 }  & & & \\\cline{2-2}\cline{3-3}\cline{5-5}
16 & \multirow{1}{*}{ 3.37 }  & \multirow{5}{*}{ 15.54 }  & & \multirow{15}{*}{ 48.71 }  \\\cline{2-2}
17 & \multirow{1}{*}{ 3.45 }  & & & \\\cline{2-2}
18 & \multirow{1}{*}{ 3.01 }  & & & \\\cline{2-2}
19 & \multirow{1}{*}{ 2.97 }  & & & \\\cline{2-2}
20 & \multirow{1}{*}{ 2.75 }  & & & \\\cline{2-2}\cline{3-3}\cline{4-4}
21 & \multirow{1}{*}{ 3.26 }  & \multirow{5}{*}{ 16.97 }  & \multirow{10}{*}{ 33.17 }  & \\\cline{2-2}
22 & \multirow{1}{*}{ 3.70 }  & & & \\\cline{2-2}
23 & \multirow{1}{*}{ 3.20 }  & & & \\\cline{2-2}
24 & \multirow{1}{*}{ 3.22 }  & & & \\\cline{2-2}
25 & \multirow{1}{*}{ 3.59 }  & & & \\\cline{2-2}\cline{3-3}
26 & \multirow{1}{*}{ 3.51 }  & \multirow{5}{*}{ 16.20 }  & & \\\cline{2-2}
27 & \multirow{1}{*}{ 3.12 }  & & & \\\cline{2-2}
28 & \multirow{1}{*}{ 3.26 }  & & & \\\cline{2-2}
29 & \multirow{1}{*}{ 3.70 }  & & & \\\cline{2-2}
30 & \multirow{1}{*}{ 2.62 }  & & & \\\cline{2-2}\cline{3-3}\cline{4-4}\cline{5-5}
\hline\end{tabular}
\end{center}
\label{tab:min}
\end{table}

\FloatBarrier
\subsection{Histograms of activity along months of the year}
	Activity percentages of the months along the year from LAD list messages. Activity is concentrated in Jun-Aug for MET and LAD, and in Dec-Mar for CPP, LAU and LAD. These observations fit academic calendars, vacations and end-of-year holidays.

\begin{table}[!h]
	\caption{LAU activity along the months of the year.}
	\footnotesize
	\begin{center}
\begin{tabular}{l || c | c | c | c | c }\hline
 & m. & b. & t. & q. & s. \\\hline
Jan & \multirow{1}{*}{ 10.22 }  & \multirow{2}{*}{ 19.56 }  & \multirow{3}{*}{ 28.24 }  & \multirow{4}{*}{ 35.09 }  & \multirow{6}{*}{ 49.16 }  \\\cline{2-2}
Fev & \multirow{1}{*}{ 9.34 }  & & & & \\\cline{2-2}\cline{3-3}
Mar & \multirow{1}{*}{ 8.67 }  & \multirow{2}{*}{ 15.53 }  & & & \\\cline{2-2}\cline{4-4}
Apr & \multirow{1}{*}{ 6.86 }  & & \multirow{3}{*}{ 20.93 }  & & \\\cline{2-2}\cline{3-3}\cline{5-5}
Mai & \multirow{1}{*}{ 7.28 }  & \multirow{2}{*}{ 14.07 }  & & \multirow{4}{*}{ 30.36 }  & \\\cline{2-2}
Jun & \multirow{1}{*}{ 6.80 }  & & & & \\\cline{2-2}\cline{3-3}\cline{4-4}\cline{6-6}
Jul & \multirow{1}{*}{ 8.97 }  & \multirow{2}{*}{ 16.29 }  & \multirow{3}{*}{ 24.47 }  & & \multirow{6}{*}{ 50.84 }  \\\cline{2-2}
Ago & \multirow{1}{*}{ 7.32 }  & & & & \\\cline{2-2}\cline{3-3}\cline{5-5}
Set & \multirow{1}{*}{ 8.18 }  & \multirow{2}{*}{ 16.25 }  & & \multirow{4}{*}{ 34.55 }  & \\\cline{2-2}\cline{4-4}
Out & \multirow{1}{*}{ 8.06 }  & & \multirow{3}{*}{ 26.36 }  & & \\\cline{2-2}\cline{3-3}
Nov & \multirow{1}{*}{ 7.64 }  & \multirow{2}{*}{ 18.30 }  & & & \\\cline{2-2}
Dez & \multirow{1}{*}{ 10.66 }  & & & & \\\cline{2-2}\cline{3-3}\cline{4-4}\cline{5-5}\cline{6-6}
\hline\end{tabular}
\end{center}

\label{tab:min2}
\end{table}
\begin{table}[!h]
	\caption{LAD activity along the months of the year.}
	\footnotesize
	\begin{center}
\begin{tabular}{| l || c | c | c | c | c |}\hline
 & m. & b. & t. & q. & s. \\\hline
Jan & \multirow{1}{*}{ 11.24 }  & \multirow{2}{*}{ 18.51 }  & \multirow{3}{*}{ 26.46 }  & \multirow{4}{*}{ 36.07 }  & \multirow{6}{*}{ 57.96 }  \\\cline{2-2}
Fev & \multirow{1}{*}{ 7.26 }  & & & & \\\cline{2-2}\cline{3-3}
Mar & \multirow{1}{*}{ 7.95 }  & \multirow{2}{*}{ 17.56 }  & & & \\\cline{2-2}\cline{4-4}
Apr & \multirow{1}{*}{ 9.61 }  & & \multirow{3}{*}{ 31.50 }  & & \\\cline{2-2}\cline{3-3}\cline{5-5}
Mai & \multirow{1}{*}{ 8.94 }  & \multirow{2}{*}{ 21.89 }  & & \multirow{4}{*}{ 37.56 }  & \\\cline{2-2}
Jun & \multirow{1}{*}{ 12.95 }  & & & & \\\cline{2-2}\cline{3-3}\cline{4-4}\cline{6-6}
Jul & \multirow{1}{*}{ 9.03 }  & \multirow{2}{*}{ 15.67 }  & \multirow{3}{*}{ 22.30 }  & & \multirow{6}{*}{ 42.04 }  \\\cline{2-2}
Ago & \multirow{1}{*}{ 6.64 }  & & & & \\\cline{2-2}\cline{3-3}\cline{5-5}
Set & \multirow{1}{*}{ 6.63 }  & \multirow{2}{*}{ 12.38 }  & & \multirow{4}{*}{ 26.37 }  & \\\cline{2-2}\cline{4-4}
Out & \multirow{1}{*}{ 5.75 }  & & \multirow{3}{*}{ 19.74 }  & & \\\cline{2-2}\cline{3-3}
Nov & \multirow{1}{*}{ 7.61 }  & \multirow{2}{*}{ 13.99 }  & & & \\\cline{2-2}
Dez & \multirow{1}{*}{ 6.38 }  & & & & \\\cline{2-2}\cline{3-3}\cline{4-4}\cline{5-5}\cline{6-6}
\hline\end{tabular}
\end{center}
\label{tab:min2}
\end{table}
\begin{table}[!h]
	\caption{MET activity along the months of the year.}
	\footnotesize
	\begin{center}
\begin{tabular}{| l || c | c | c | c | c |}\hline
 & m. & b. & t. & q. & s. \\\hline
Jan & \multirow{1}{*}{ 4.87 }  & \multirow{2}{*}{ 11.00 }  & \multirow{3}{*}{ 16.89 }  & \multirow{4}{*}{ 23.30 }  & \multirow{6}{*}{ 47.70 }  \\\cline{2-2}
Fev & \multirow{1}{*}{ 6.13 }  & & & & \\\cline{2-2}\cline{3-3}
Mar & \multirow{1}{*}{ 5.89 }  & \multirow{2}{*}{ 12.30 }  & & & \\\cline{2-2}\cline{4-4}
Apr & \multirow{1}{*}{ 6.41 }  & & \multirow{3}{*}{ 30.81 }  & & \\\cline{2-2}\cline{3-3}\cline{5-5}
Mai & \multirow{1}{*}{ 10.45 }  & \multirow{2}{*}{ 24.40 }  & & \multirow{4}{*}{ 47.87 }  & \\\cline{2-2}
Jun & \multirow{1}{*}{ 13.95 }  & & & & \\\cline{2-2}\cline{3-3}\cline{4-4}\cline{6-6}
Jul & \multirow{1}{*}{ 13.24 }  & \multirow{2}{*}{ 23.47 }  & \multirow{3}{*}{ 31.21 }  & & \multirow{6}{*}{ 52.30 }  \\\cline{2-2}
Ago & \multirow{1}{*}{ 10.22 }  & & & & \\\cline{2-2}\cline{3-3}\cline{5-5}
Set & \multirow{1}{*}{ 7.75 }  & \multirow{2}{*}{ 16.79 }  & & \multirow{4}{*}{ 28.83 }  & \\\cline{2-2}\cline{4-4}
Out & \multirow{1}{*}{ 9.04 }  & & \multirow{3}{*}{ 21.09 }  & & \\\cline{2-2}\cline{3-3}
Nov & \multirow{1}{*}{ 7.45 }  & \multirow{2}{*}{ 12.05 }  & & & \\\cline{2-2}
Dez & \multirow{1}{*}{ 4.59 }  & & & & \\\cline{2-2}\cline{3-3}\cline{4-4}\cline{5-5}\cline{6-6}
\hline\end{tabular}
\end{center}
\label{tab:min2}
\end{table}

\begin{table}[!h]
	\caption{CPP activity along the months of the year.}
	\footnotesize
	\begin{center}
\begin{tabular}{| l || c | c | c | c | c |}\hline
 & m. & b. & t. & q. & s. \\\hline
Jan & \multirow{1}{*}{ 8.70 }  & \multirow{2}{*}{ 17.00 }  & \multirow{3}{*}{ 27.23 }  & \multirow{4}{*}{ 36.49 }  & \multirow{6}{*}{ 54.27 }  \\\cline{2-2}
Fev & \multirow{1}{*}{ 8.29 }  & & & & \\\cline{2-2}\cline{3-3}
Mar & \multirow{1}{*}{ 10.23 }  & \multirow{2}{*}{ 19.49 }  & & & \\\cline{2-2}\cline{4-4}
Apr & \multirow{1}{*}{ 9.26 }  & & \multirow{3}{*}{ 27.03 }  & & \\\cline{2-2}\cline{3-3}\cline{5-5}
Mai & \multirow{1}{*}{ 9.41 }  & \multirow{2}{*}{ 17.78 }  & & \multirow{4}{*}{ 33.46 }  & \\\cline{2-2}
Jun & \multirow{1}{*}{ 8.37 }  & & & & \\\cline{2-2}\cline{3-3}\cline{4-4}\cline{6-6}
Jul & \multirow{1}{*}{ 8.70 }  & \multirow{2}{*}{ 15.68 }  & \multirow{3}{*}{ 22.94 }  & & \multirow{6}{*}{ 45.73 }  \\\cline{2-2}
Ago & \multirow{1}{*}{ 6.98 }  & & & & \\\cline{2-2}\cline{3-3}\cline{5-5}
Set & \multirow{1}{*}{ 7.26 }  & \multirow{2}{*}{ 15.36 }  & & \multirow{4}{*}{ 30.06 }  & \\\cline{2-2}\cline{4-4}
Out & \multirow{1}{*}{ 8.10 }  & & \multirow{3}{*}{ 22.80 }  & & \\\cline{2-2}\cline{3-3}
Nov & \multirow{1}{*}{ 7.89 }  & \multirow{2}{*}{ 14.69 }  & & & \\\cline{2-2}
Dez & \multirow{1}{*}{ 6.81 }  & & & & \\\cline{2-2}\cline{3-3}\cline{4-4}\cline{5-5}\cline{6-6}
\hline\end{tabular}
\end{center}
\label{tab:min2}
\end{table}


\section{Fraction of participants in each Erd\"os Sector along the timeline}\label{sec:frac}

\section{Fraction of participants in each Erd\"os Sector along the timeline}\label{sec:pcat}
\subsection{Betweenness, clustering and degree}
\subsection{Betweenness, clustering, degrees and strengths}
\subsection{Betweenness, clustering, degrees, strengths and symmetry measures}

\nocite{*}
\bibliography{supportingInformation}% Produces the bibliography via BibTeX.
\end{document}
%
% ****** End of file aipsamp.tex ******


