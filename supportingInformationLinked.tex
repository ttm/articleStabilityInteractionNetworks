% ****** Start of file aipsamp.tex ******
%
%   This file is part of the AIP files in the AIP distribution for REVTeX 4.
%   Version 4.1 of REVTeX, October 2009
%
%   Copyright (c) 2009 American Institute of Physics.

% Use this file as a source of example code for your aip document.
% Use the file aiptemplate.tex as a template for your document.

\newcommand{\wbalsup}[1] {
	  This is the Wikibook about LaTeX 
	    supported by #1}

\newcommand{\minput}[1] {
	  /home/r/repos/percolation/tests/tables/#1}
\newcommand{\tinput}[1] {
%	\input{/home/r/repos/percolation/tests/tables/#1}
	\input{/home/r/repos/documentation/tables/#1}
}
\documentclass[%
 aip,
 jmp,%
 amsmath,amssymb,
%preprint,%
 reprint,%
 floatfix,
%author-year,%
%author-numerical,%
]{revtex4-1}

\usepackage{longtable}
\usepackage{graphicx}% Include figure files
\usepackage{grffile}
\usepackage{dcolumn}% Align table columns on decimal point
\usepackage{bm}% bold math
%\usepackage[mathlines]{lineno}% Enable numbering of text and display math
%\linenumbers\relax % Commence numbering lines
\usepackage{multirow}
\usepackage{color} % for the notes
\usepackage{etex}
\reserveinserts{58}
%\usepackage{morefloats}
\usepackage{hyperref}
\usepackage[usenames,dvipsnames]{xcolor}
\usepackage{amsmath}
\hypersetup{
        colorlinks,
        linkcolor={red!50!black},
        citecolor={blue!50!black},
        urlcolor={blue!80!black}
}
%\usepackage{placeins}
\usepackage{xr}
\externaldocument{paper}
\usepackage[section] {placeins}

\newcommand{\beginsupplement}{%
%	\setcounter{table}{0}
	\renewcommand{\thesection}{S\Roman{section}}%
	\setcounter{table}{0}
	\renewcommand{\thetable}{S\arabic{table}}%
	\setcounter{figure}{0}
	\renewcommand{\thefigure}{S\arabic{figure}}%
}

\beginsupplement

\begin{document}

\preprint{XXXXX (preprint)}

%\title[Evolution of interaction networks]{On the evolution of interaction networks: primitive typology of vertex, prominence of measures and activity statistics}% Force line breaks with \\
%\title[Evolution of interaction networks]{On the evolution of interaction networks: a primitive typology of vertex}% Force line breaks with \\
%\title[Interaction networks stability: SUPPORTING INFORMATION]{Time stability in human interaction networks: primitive typology of vertex, prominence of measures and time activity statistics (SUPPORTING INFORMATION)}% Force line breaks with \\

\title[Interaction networks stability (Supporting Information)]{Temporal stability in human interaction networks (Supporting Information document)}% Force line breaks with \\


\author{Renato Fabbri}%
\homepage{http://ifsc.usp.br/~fabbri/}
\email{fabbri@usp.br}
\affiliation{ 
	S\~ao Carlos Institute of Physics, University of S\~ao Paulo (IFSC/USP),
	PO Box 369, 13560-970, S\~ao Carlos, SP, Brazil %\\This line break forced with \textbackslash\textbackslash
}
%
%\author{Vilson V. da Silva Jr.}
%\homepage{http://automata.cc/}
%\email{vilson@void.cc}
%\altaffiliation[Also at ]{IFSC-USP}%Lines break automatically or can be forced with \\
%
\author{Ricardo Fabbri}
\homepage{http://www.lems.brown.edu/~rfabbri/}
\email{rfabbri@iprj.uerj.br}
\altaffiliation{
	Instituto Polit\'ecnico, Universidade Estadual do Rio de Janeiro (IPRJ)
}%Lines break automatically or can be forced with \\

\author{Deborah C. Antunes}
\homepage{http://lattes.cnpq.br/1065956470701739}
\email{deborahantunes@gmail.com}
\altaffiliation{
	Curso de Psicologia, Universidade Federal do Cer\'a (UFC)
}%Lines break automatically or can be forced with \\

\author{Marilia M. Pisani}
\homepage{http://lattes.cnpq.br/6738980149860322}
\email{marilia.m.pisani@gmail.com}
\altaffiliation{
       %Centro de Ciências Naturais e Humanas, Universidade Federal do ABC (CCNH/UFABC)
}%Lines break automatically or can be forced with \\

%
%%\author{Luciano da Fontoura Costa}
%%  \homepage{http://cyvision.ifsc.usp.br/~luciano/}
%%  \email{ldfcosta@gmail.com}
%  \altaffiliation[Also at ]{IFSC-USP}%Lines break automatically or can be forced with \\
%
\author{Leonardo P. Maia}
  \homepage{http://www.ifsc.usp.br/~lpmaia/}
  \email{lpmaia@ifsc.usp.br }
 \altaffiliation[Also at ]{IFSC-USP}%Lines break automatically or can be forced with \\


\author{Osvaldo N. Oliveira Jr.}
  \homepage{www.polimeros.ifsc.usp.br/professors/professor.php?id=4}
  \email{chu@ifsc.usp.br}
 \altaffiliation[Also at ]{IFSC-USP}%Lines break automatically or can be forced with \\

%\author{Renato Fabbri}%
% \homepage{http://ifsc.usp.br/~fabbri/}
% \email{fabbri@usp.br}
%  \affiliation{ 
%S\~ao Carlos Institute of Physics, University of S\~ao Paulo (IFSC/USP)%\\This line break forced with \textbackslash\textbackslash
%}
%
%%\author{Vilson V. da Silva Jr.}
%%  \homepage{http://automata.cc/}
%%  \email{vilson@void.cc}
%%  \altaffiliation[Also at ]{IFSC-USP}%Lines break automatically or can be forced with \\
%%
%\author{Ricardo Fabbri}
%  \homepage{http://www.lems.brown.edu/~rfabbri/}
%  \email{rfabbri@iprj.uerj.br}
% \altaffiliation{
%Instituto Polit\'ecnico, Universidade Estadual do Rio de Janeiro (IPRJ)
%}%Lines break automatically or can be forced with \\
%
%\author{Deborah C. Antunes}
%  \homepage{http://lattes.cnpq.br/1065956470701739}
%  \email{deborahantunes@gmail.com}
%  \altaffiliation{
%Curso de Psicologia, Universidade Federal do Cer\'a (UFC)
%}%Lines break automatically or can be forced with \\
%
%\author{Marilia M. Pisani}
%  \homepage{http://lattes.cnpq.br/6738980149860322}
%  \email{marilia.m.pisani@gmail.com}
% \altaffiliation{
%Centro de Ci\^encias Naturais e Humanas, Universidade Federal do ABC (CCNH/UFABC)
%}%Lines break automatically or can be forced with \\
%
%%
%%%\author{Luciano da Fontoura Costa}
%%%  \homepage{http://cyvision.ifsc.usp.br/~luciano/}
%%%  \email{ldfcosta@gmail.com}
%%  \altaffiliation[Also at ]{IFSC-USP}%Lines break automatically or can be forced with \\
%\author{Leonardo Paulo Maia}
%  \homepage{http://www.ifsc.usp.br/~lpmaia/}
%  \email{lpmaia@ifsc.usp.br }
% \altaffiliation[Also at ]{IFSC-USP}%Lines break automatically or can be forced with \\
%
%
%
%\author{Osvaldo N. Oliveira Jr.}
%  \homepage{www.polimeros.ifsc.usp.br/professors/professor.php?id=4}
%  \email{chu@ifsc.usp.br}
% \altaffiliation[Also at ]{IFSC-USP}%Lines break automatically or can be forced with \\


\date{\today}% It is always \today, today,
             %  but any date may be explicitly specified

\maketitle

\tableofcontents


\vfill
\newpage

This Supporting Information document holds circular statistics and histograms of activity along time in Section~\ref{sec:time},
the fraction of vertices in the peripheral, intermediary and hub sectors in Section~\ref{si:frac}
and the combination of basic topological measures into principal components with greater variance in Section~\ref{si:pcat}.
There is a focus on email list interaction networks for benchmarking and
Section~\ref{si:ext} reinforces the results with the analysis of networks from Facebook, Twitter and Participabr.
More context (e.g. methods, discussion, data and scripts) is given in the main document~\cite{tpaper}
to which this current document supplies supporting information.

\section{RDF units}\label{sec:rdf}
% n triples, subs preds objs
% n classes, n individuos
%seconds & --//--  & 3.13  & 0.99  & 9070.17  & 1.28 \\\hline
minutes & --//--  & 3.60  & 1.00  & 205489.40  & 1.22 \\\hline
hours & -9.61  & 1.52  & 0.68  & 4.36  & 9.77 \\\hline
weekdays & -0.03  & 2.03  & 0.87  & 29.28  & 1.72 \\\hline
month days & -0.07  & 2.94  & 0.99  & 2754.16  & 2.21 \\\hline
months & -0.56  & 2.14  & 0.90  & 44.00  & 2.25 \\\hline

% 
\tinput{rdfUnits}
\section{Overview of the systems}\label{sec:over}
% n participants, n relations, n interactions
% date snap or from to
% provenance: fb, tw, irc, email

% retoma secoes anteriores, renderizando as tabelas quando cabivel
\bibliography{supportingInformation}% Produces the bibliography via BibTeX.
\end{document}
%
\section{Time activity in different scales}\label{sec:time}
This section presents information derived from the theory presented in Section~\ref{sec:mtime}
for supporting the results in Section~\ref{constDisc}.

\subsection{Time circular measures}\label{si:circ}
The metrics with which we report measurements and results
of activity along time are the rescaled circular mean $\theta_\mu'$,
the standard deviation $S(z)$, the variance $Var(z)$, the circular dispersion $\delta(z)$
and the ration between the lowest $b_l$ and the highest $b_h$ incidences $\frac{b_l}{b_h}$ at each time scale.
Also, the mean $\mu_{\frac{b_l'}{b_h'}}$ and 
the standard deviation $\sigma_{\frac{b_l'}{b_h'}}$ 
of the relation between the minimum $b_l'$ and the maximum $b_h'$ incidences
are given for 1000 uniform distribution simulations within the 
same number of bins and with the same number of samples~\footnote{Numpy version 1.8.2, ``random.randint'' function, was used for simulations, algorithms in \url{https://pypi.python.org/pypi/gmane}}.
Greater dispersion is found on the scales of seconds and minutes, followed
by days of the month.
Greater localization is found in the scale of hours of the day, followed by weekdays and months.

\begin{table*}[!h]
	\caption{LAU circular measurements.}
\begin{center}
    \begin{tabular}{ |l|| c|c|c|c|c||c|c| }
        \hline
scale & $\theta_\mu'$ & $S(z)$ & $Var(z)$ & $\delta(z)$ & $\frac{b_l}{b_h}$ & $ \mu_{\frac{b_l'}{b_h'}} $ & $ \sigma_{\frac{b_l'}{b_h'}} $ \\ \hline
	seconds & --//--  & 3.31  & 1.00  & 29337.65  & 0.78  & 0.78  & 0.03 \\\hline
minutes & --//--  & 3.13  & 0.99  & 8879.19  & 0.76  & 0.78  & 0.03 \\\hline
hours & -8.76  & 1.56  & 0.71  & 4.92  & 0.12  & 0.87  & 0.02 \\\hline
weekdays & -0.21  & 2.14  & 0.90  & 45.41  & 0.62  & 0.95  & 0.01 \\\hline
month days & -0.64  & 2.76  & 0.98  & 1001.75  & 0.67  & 0.85  & 0.02 \\\hline
months & 3.55  & 2.30  & 0.93  & 94.53  & 0.64  & 0.92  & 0.02 \\\hline

    \end{tabular}
\end{center}
\label{tab:circLau}
\end{table*}
\begin{table*}[!h]
	\caption{LAD circular measurements.}
\begin{center}
    \begin{tabular}{ |l|| c|c|c|c|c||c|c| }
        \hline
scale & $\theta_\mu'$ & $S(z)$ & $Var(z)$ & $\delta(z)$ & $\frac{b_l}{b_h}$ & $ \mu_{\frac{b_l'}{b_h'}} $ & $ \sigma_{\frac{b_l'}{b_h'}} $ \\ \hline
	seconds & --//--  & 3.13  & 0.99  & 9070.17  & 1.28 \\\hline
minutes & --//--  & 3.60  & 1.00  & 205489.40  & 1.22 \\\hline
hours & -9.61  & 1.52  & 0.68  & 4.36  & 9.77 \\\hline
weekdays & -0.03  & 2.03  & 0.87  & 29.28  & 1.72 \\\hline
month days & -0.07  & 2.94  & 0.99  & 2754.16  & 2.21 \\\hline
months & -0.56  & 2.14  & 0.90  & 44.00  & 2.25 \\\hline

    \end{tabular}
\end{center}
\label{tab:circLad}
\end{table*}
\begin{table*}[!h]
	\caption{MET circular measurements.}
\begin{center}
    \begin{tabular}{ |l|| c|c|c|c|c||c|c| }
        \hline
scale & $\theta_\mu'$ & $S(z)$ & $Var(z)$ & $\delta(z)$ & $\frac{b_l}{b_h}$ & $ \mu_{\frac{b_l'}{b_h'}} $ & $ \sigma_{\frac{b_l'}{b_h'}} $ \\ \hline
	seconds & --//--  & 3.06  & 0.99  & 5910.47  & 1.27 \\\hline
minutes & --//--  & 3.14  & 0.99  & 9696.29  & 1.34 \\\hline
hours & -9.20  & 1.35  & 0.60  & 2.76  & 19.26 \\\hline
weekdays & -0.27  & 1.86  & 0.82  & 13.82  & 2.89 \\\hline
month days & 0.14  & 2.49  & 0.95  & 237.27  & 1.93 \\\hline
months & -2.92  & 1.73  & 0.78  & 9.20  & 3.04 \\\hline

    \end{tabular}
\end{center}
\label{tab:circMet}
\end{table*}
\begin{table*}[!h]
	\caption{CPP circular measurements.}
\begin{center}
    \begin{tabular}{ |l|| c|c|c|c|c||c|c| }
        \hline
scale & $\theta_\mu'$ & $S(z)$ & $Var(z)$ & $\delta(z)$ & $\frac{b_l}{b_h}$ & $ \mu_{\frac{b_l'}{b_h'}} $ & $ \sigma_{\frac{b_l'}{b_h'}}$ \\ \hline
	seconds & --//--  & 3.31  & 1.00  & 28205.46  & 1.26 \\\hline
minutes & --//--  & 3.18  & 0.99  & 12275.59  & 1.27 \\\hline
hours & -9.39  & 1.48  & 0.67  & 3.91  & 11.18 \\\hline
weekdays & -0.17  & 1.83  & 0.81  & 12.66  & 2.59 \\\hline
month days & -0.31  & 3.14  & 0.99  & 9541.67  & 1.93 \\\hline
months & 0.15  & 2.34  & 0.93  & 115.49  & 1.50 \\\hline

    \end{tabular}
\end{center}
\label{tab:circCPP}
\end{table*}

\FloatBarrier
\subsection{Time histograms}
\subsubsection{Histograms of activity along the hours of the day}\label{si:hours}

Higher activity was observed between noon and 6pm, followed by the time period between 6pm and midnight.
Around 2/3 of the whole activity takes place from noon to midnight.
The activity peak occurs around midday, with a slight skew toward one hour before noon.
\begin{table}[!h]
	\caption{LAU activity along the hours of the day.}
	\footnotesize
	\begin{center}
\begin{tabular}{l || c | c | c | c | c | c |}\hline
 & 1h & 2h & 3h & 4h & 6h & 12h \\\hline
0h & \multirow{1}{*}{ 3.58 }  & \multirow{2}{*}{ 5.80 }  & \multirow{3}{*}{ 7.43 }  & \multirow{4}{*}{ 8.49 }  & \multirow{6}{*}{ 10.14 }  & \multirow{12}{*}{ 36.88 }  \\\cline{1-1}
1h & \multirow{1}{*}{ 2.22 }  & & & & & \\\cline{1-1}\cline{2-2}
2h & \multirow{1}{*}{ 1.63 }  & \multirow{2}{*}{ 2.69 }  & & & & \\\cline{1-1}\cline{3-3}
3h & \multirow{1}{*}{ 1.06 }  & & \multirow{3}{*}{ 2.72 }  & & & \\\cline{1-1}\cline{2-2}\cline{4-4}
4h & \multirow{1}{*}{ 0.84 }  & \multirow{2}{*}{ 1.66 }  & & \multirow{4}{*}{ 5.20 }  & & \\\cline{1-1}
5h & \multirow{1}{*}{ 0.82 }  & & & & & \\\cline{1-1}\cline{2-2}\cline{3-3}\cline{5-5}
6h & \multirow{1}{*}{ 1.17 }  & \multirow{2}{*}{ 3.54 }  & \multirow{3}{*}{ 7.07 }  & & \multirow{6}{*}{ 26.74 }  & \\\cline{1-1}
7h & \multirow{1}{*}{ 2.37 }  & & & & & \\\cline{1-1}\cline{2-2}\cline{4-4}
8h & \multirow{1}{*}{ 3.53 }  & \multirow{2}{*}{ 9.57 }  & & \multirow{4}{*}{ 23.20 }  & & \\\cline{1-1}\cline{3-3}
9h & \multirow{1}{*}{ 6.04 }  & & \multirow{3}{*}{ 19.67 }  & & & \\\cline{1-1}\cline{2-2}
10h & \multirow{1}{*}{ 6.83 }  & \multirow{2}{*}{ 13.62 }  & & & & \\\cline{1-1}
11h & \multirow{1}{*}{ 6.79 }  & & & & & \\\cline{1-1}\cline{2-2}\cline{3-3}\cline{4-4}\cline{5-5}\cline{6-6}
12h & \multirow{1}{*}{ 6.11 }  & \multirow{2}{*}{ 12.36 }  & \multirow{3}{*}{ 18.75 }  & \multirow{4}{*}{ 24.68 }  & \multirow{6}{*}{ 35.66 }  & \multirow{12}{*}{ 63.12 }  \\\cline{1-1}
13h & \multirow{1}{*}{ 6.26 }  & & & & & \\\cline{1-1}\cline{2-2}
14h & \multirow{1}{*}{ 6.38 }  & \multirow{2}{*}{ 12.31 }  & & & & \\\cline{1-1}\cline{3-3}
15h & \multirow{1}{*}{ 5.93 }  & & \multirow{3}{*}{ 16.91 }  & & & \\\cline{1-1}\cline{2-2}\cline{4-4}
16h & \multirow{1}{*}{ 5.52 }  & \multirow{2}{*}{ 10.98 }  & & \multirow{4}{*}{ 20.73 }  & & \\\cline{1-1}
17h & \multirow{1}{*}{ 5.46 }  & & & & & \\\cline{1-1}\cline{2-2}\cline{3-3}\cline{5-5}
18h & \multirow{1}{*}{ 5.23 }  & \multirow{2}{*}{ 9.75 }  & \multirow{3}{*}{ 14.30 }  & & \multirow{6}{*}{ 27.46 }  & \\\cline{1-1}
19h & \multirow{1}{*}{ 4.52 }  & & & & & \\\cline{1-1}\cline{2-2}\cline{4-4}
20h & \multirow{1}{*}{ 4.55 }  & \multirow{2}{*}{ 8.97 }  & & \multirow{4}{*}{ 17.71 }  & & \\\cline{1-1}\cline{3-3}
21h & \multirow{1}{*}{ 4.42 }  & & \multirow{3}{*}{ 13.16 }  & & & \\\cline{1-1}\cline{2-2}
22h & \multirow{1}{*}{ 4.51 }  & \multirow{2}{*}{ 8.74 }  & & & & \\\cline{1-1}
23h & \multirow{1}{*}{ 4.23 }  & & & & & \\\cline{1-1}\cline{2-2}\cline{3-3}\cline{4-4}\cline{5-5}\cline{6-6}
\end{tabular}
\end{center}
\end{table}

\begin{table}[!h]
	\caption{LAD activity along the hours of the day.}
	\footnotesize
	\begin{center}
\begin{tabular}{l || c | c | c | c | c | c |}\hline
 & 1h & 2h & 3h & 4h & 6h & 12h \\\hline
0h & \multirow{1}{*}{ 4.01 }  & \multirow{2}{*}{ 6.53 }  & \multirow{3}{*}{ 8.32 }  & \multirow{4}{*}{ 9.37 }  & \multirow{6}{*}{ 10.78 }  & \multirow{12}{*}{ 33.11 }  \\\cline{1-1}
1h & \multirow{1}{*}{ 2.52 }  & & & & & \\\cline{1-1}\cline{2-2}
2h & \multirow{1}{*}{ 1.79 }  & \multirow{2}{*}{ 2.84 }  & & & & \\\cline{1-1}\cline{3-3}
3h & \multirow{1}{*}{ 1.06 }  & & \multirow{3}{*}{ 2.46 }  & & & \\\cline{1-1}\cline{2-2}\cline{4-4}
4h & \multirow{1}{*}{ 0.75 }  & \multirow{2}{*}{ 1.40 }  & & \multirow{4}{*}{ 3.81 }  & & \\\cline{1-1}
5h & \multirow{1}{*}{ 0.66 }  & & & & & \\\cline{1-1}\cline{2-2}\cline{3-3}\cline{5-5}
6h & \multirow{1}{*}{ 0.85 }  & \multirow{2}{*}{ 2.41 }  & \multirow{3}{*}{ 5.36 }  & & \multirow{6}{*}{ 22.33 }  & \\\cline{1-1}
7h & \multirow{1}{*}{ 1.56 }  & & & & & \\\cline{1-1}\cline{2-2}\cline{4-4}
8h & \multirow{1}{*}{ 2.95 }  & \multirow{2}{*}{ 7.61 }  & & \multirow{4}{*}{ 19.93 }  & & \\\cline{1-1}\cline{3-3}
9h & \multirow{1}{*}{ 4.66 }  & & \multirow{3}{*}{ 16.98 }  & & & \\\cline{1-1}\cline{2-2}
10h & \multirow{1}{*}{ 5.92 }  & \multirow{2}{*}{ 12.32 }  & & & & \\\cline{1-1}
11h & \multirow{1}{*}{ 6.40 }  & & & & & \\\cline{1-1}\cline{2-2}\cline{3-3}\cline{4-4}\cline{5-5}\cline{6-6}
12h & \multirow{1}{*}{ 6.41 }  & \multirow{2}{*}{ 12.53 }  & \multirow{3}{*}{ 18.85 }  & \multirow{4}{*}{ 24.82 }  & \multirow{6}{*}{ 37.24 }  & \multirow{12}{*}{ 66.89 }  \\\cline{1-1}
13h & \multirow{1}{*}{ 6.12 }  & & & & & \\\cline{1-1}\cline{2-2}
14h & \multirow{1}{*}{ 6.32 }  & \multirow{2}{*}{ 12.29 }  & & & & \\\cline{1-1}\cline{3-3}
15h & \multirow{1}{*}{ 5.97 }  & & \multirow{3}{*}{ 18.39 }  & & & \\\cline{1-1}\cline{2-2}\cline{4-4}
16h & \multirow{1}{*}{ 6.40 }  & \multirow{2}{*}{ 12.42 }  & & \multirow{4}{*}{ 23.44 }  & & \\\cline{1-1}
17h & \multirow{1}{*}{ 6.02 }  & & & & & \\\cline{1-1}\cline{2-2}\cline{3-3}\cline{5-5}
18h & \multirow{1}{*}{ 5.99 }  & \multirow{2}{*}{ 11.02 }  & \multirow{3}{*}{ 15.65 }  & & \multirow{6}{*}{ 29.65 }  & \\\cline{1-1}
19h & \multirow{1}{*}{ 5.03 }  & & & & & \\\cline{1-1}\cline{2-2}\cline{4-4}
20h & \multirow{1}{*}{ 4.63 }  & \multirow{2}{*}{ 9.22 }  & & \multirow{4}{*}{ 18.63 }  & & \\\cline{1-1}\cline{3-3}
21h & \multirow{1}{*}{ 4.59 }  & & \multirow{3}{*}{ 14.00 }  & & & \\\cline{1-1}\cline{2-2}
22h & \multirow{1}{*}{ 4.88 }  & \multirow{2}{*}{ 9.41 }  & & & & \\\cline{1-1}
23h & \multirow{1}{*}{ 4.53 }  & & & & & \\\cline{1-1}\cline{2-2}\cline{3-3}\cline{4-4}\cline{5-5}\cline{6-6}
\end{tabular}
\end{center}
\end{table}

\begin{table}[!h]
	\caption{MET activity along the hours of the day.}
	\footnotesize
	\begin{center}
\begin{tabular}{| l || c | c | c | c | c | c |}\hline
 & 1h & 2h & 3h & 4h & 6h & 12h \\\hline
0h & \multirow{1}{*}{ 2.87 }  & \multirow{2}{*}{ 4.64 }  & \multirow{3}{*}{ 5.67 }  & \multirow{4}{*}{ 6.31 }  & \multirow{6}{*}{ 7.15 }  & \multirow{12}{*}{ 29.33 }  \\\cline{2-2}
1h & \multirow{1}{*}{ 1.77 }  & & & & & \\\cline{2-2}\cline{3-3}
2h & \multirow{1}{*}{ 1.04 }  & \multirow{2}{*}{ 1.67 }  & & & & \\\cline{2-2}\cline{4-4}
3h & \multirow{1}{*}{ 0.64 }  & & \multirow{3}{*}{ 1.48 }  & & & \\\cline{2-2}\cline{3-3}\cline{5-5}
4h & \multirow{1}{*}{ 0.47 }  & \multirow{2}{*}{ 0.85 }  & & \multirow{4}{*}{ 2.89 }  & & \\\cline{2-2}
5h & \multirow{1}{*}{ 0.38 }  & & & & & \\\cline{2-2}\cline{3-3}\cline{4-4}\cline{6-6}
6h & \multirow{1}{*}{ 0.72 }  & \multirow{2}{*}{ 2.04 }  & \multirow{3}{*}{ 4.71 }  & & \multirow{6}{*}{ 22.18 }  & \\\cline{2-2}
7h & \multirow{1}{*}{ 1.33 }  & & & & & \\\cline{2-2}\cline{3-3}\cline{5-5}
8h & \multirow{1}{*}{ 2.67 }  & \multirow{2}{*}{ 7.07 }  & & \multirow{4}{*}{ 20.14 }  & & \\\cline{2-2}\cline{4-4}
9h & \multirow{1}{*}{ 4.40 }  & & \multirow{3}{*}{ 17.47 }  & & & \\\cline{2-2}\cline{3-3}
10h & \multirow{1}{*}{ 6.29 }  & \multirow{2}{*}{ 13.07 }  & & & & \\\cline{2-2}
11h & \multirow{1}{*}{ 6.78 }  & & & & & \\\cline{2-2}\cline{3-3}\cline{4-4}\cline{5-5}\cline{6-6}\cline{7-7}
12h & \multirow{1}{*}{ 7.33 }  & \multirow{2}{*}{ 14.41 }  & \multirow{3}{*}{ 21.50 }  & \multirow{4}{*}{ 28.65 }  & \multirow{6}{*}{ 42.22 }  & \multirow{12}{*}{ 70.67 }  \\\cline{2-2}
13h & \multirow{1}{*}{ 7.08 }  & & & & & \\\cline{2-2}\cline{3-3}
14h & \multirow{1}{*}{ 7.09 }  & \multirow{2}{*}{ 14.24 }  & & & & \\\cline{2-2}\cline{4-4}
15h & \multirow{1}{*}{ 7.14 }  & & \multirow{3}{*}{ 20.72 }  & & & \\\cline{2-2}\cline{3-3}\cline{5-5}
16h & \multirow{1}{*}{ 6.68 }  & \multirow{2}{*}{ 13.58 }  & & \multirow{4}{*}{ 24.79 }  & & \\\cline{2-2}
17h & \multirow{1}{*}{ 6.89 }  & & & & & \\\cline{2-2}\cline{3-3}\cline{4-4}\cline{6-6}
18h & \multirow{1}{*}{ 5.99 }  & \multirow{2}{*}{ 11.22 }  & \multirow{3}{*}{ 16.19 }  & & \multirow{6}{*}{ 28.44 }  & \\\cline{2-2}
19h & \multirow{1}{*}{ 5.23 }  & & & & & \\\cline{2-2}\cline{3-3}\cline{5-5}
20h & \multirow{1}{*}{ 4.98 }  & \multirow{2}{*}{ 9.34 }  & & \multirow{4}{*}{ 17.22 }  & & \\\cline{2-2}\cline{4-4}
21h & \multirow{1}{*}{ 4.37 }  & & \multirow{3}{*}{ 12.25 }  & & & \\\cline{2-2}\cline{3-3}
22h & \multirow{1}{*}{ 4.24 }  & \multirow{2}{*}{ 7.88 }  & & & & \\\cline{2-2}
23h & \multirow{1}{*}{ 3.64 }  & & & & & \\\cline{2-2}\cline{3-3}\cline{4-4}\cline{5-5}\cline{6-6}\cline{7-7}
\hline\end{tabular}
\end{center}
\end{table}

\begin{table}[!h]
	\caption{CPP activity along the hours of the day.}
	\footnotesize
	\input{tables/tabHoursCPP}
\end{table}

\FloatBarrier

\subsubsection{Histograms of activity along weekdays.}
Most notably, the pattern of activity along weekdays presents
a decrease of activity on weekend days of at least one third and at most two thirds
compared against workweek days.

\begin{table}[!h]
\begin{center}
    \begin{tabular}{ | l ||  c | c | c | c | c |   c | c |}
        \hline
        & Mon & Tue & Wed & Thu & Fri & Sat & Sun  \\ \hline
	LAU & 15.71  & 15.81  & 15.88  & 16.43  & 15.14  & 10.13  & 10.91 \\\hline
LAD & 14.92  & 17.75  & 17.01  & 15.41  & 14.21  & 10.40  & 10.31 \\\hline
MET & 17.53  & 17.54  & 16.43  & 17.06  & 17.46  & 7.92  & 6.06 \\\hline
CPP & 17.06  & 17.43  & 17.61  & 17.13  & 16.30  & 6.81  & 7.67 \\\hline

    \end{tabular}
\end{center}
\label{tab:win}
\end{table}

\FloatBarrier
\subsubsection{Histograms of activity along the days of the month}\label{si:monthdays}
The most important feature seems to be the homogeneity made explicit by the high circular dispersion in the tables of Section~\ref{si:circ}. Slightly higher activity rates are found in the beginning of the month, although not statistically significant.
\begin{table}[!h]
	\caption{LAU activity along the days of the month.}
	\footnotesize
	\begin{center}
\begin{tabular}{| l || c | c | c | c |}\hline
 & 1 day & 5 & 10 & 15 days \\\hline
1 & \multirow{1}{*}{ 3.36 }  & \multirow{5}{*}{ 16.21 }  & \multirow{10}{*}{ 33.71 }  & \multirow{15}{*}{ 50.82 }  \\\cline{2-2}
2 & \multirow{1}{*}{ 3.43 }  & & & \\\cline{2-2}
3 & \multirow{1}{*}{ 3.31 }  & & & \\\cline{2-2}
4 & \multirow{1}{*}{ 3.37 }  & & & \\\cline{2-2}
5 & \multirow{1}{*}{ 2.75 }  & & & \\\cline{2-2}\cline{3-3}
6 & \multirow{1}{*}{ 3.03 }  & \multirow{5}{*}{ 17.50 }  & & \\\cline{2-2}
7 & \multirow{1}{*}{ 3.93 }  & & & \\\cline{2-2}
8 & \multirow{1}{*}{ 3.62 }  & & & \\\cline{2-2}
9 & \multirow{1}{*}{ 3.84 }  & & & \\\cline{2-2}
10 & \multirow{1}{*}{ 3.09 }  & & & \\\cline{2-2}\cline{3-3}\cline{4-4}
11 & \multirow{1}{*}{ 3.20 }  & \multirow{5}{*}{ 17.11 }  & \multirow{10}{*}{ 34.02 }  & \\\cline{2-2}
12 & \multirow{1}{*}{ 3.40 }  & & & \\\cline{2-2}
13 & \multirow{1}{*}{ 3.67 }  & & & \\\cline{2-2}
14 & \multirow{1}{*}{ 3.71 }  & & & \\\cline{2-2}
15 & \multirow{1}{*}{ 3.14 }  & & & \\\cline{2-2}\cline{3-3}\cline{5-5}
16 & \multirow{1}{*}{ 3.08 }  & \multirow{5}{*}{ 16.91 }  & & \multirow{15}{*}{ 49.18 }  \\\cline{2-2}
17 & \multirow{1}{*}{ 3.13 }  & & & \\\cline{2-2}
18 & \multirow{1}{*}{ 3.43 }  & & & \\\cline{2-2}
19 & \multirow{1}{*}{ 3.61 }  & & & \\\cline{2-2}
20 & \multirow{1}{*}{ 3.67 }  & & & \\\cline{2-2}\cline{3-3}\cline{4-4}
21 & \multirow{1}{*}{ 3.60 }  & \multirow{5}{*}{ 15.43 }  & \multirow{10}{*}{ 32.27 }  & \\\cline{2-2}
22 & \multirow{1}{*}{ 3.42 }  & & & \\\cline{2-2}
23 & \multirow{1}{*}{ 2.80 }  & & & \\\cline{2-2}
24 & \multirow{1}{*}{ 2.64 }  & & & \\\cline{2-2}
25 & \multirow{1}{*}{ 2.97 }  & & & \\\cline{2-2}\cline{3-3}
26 & \multirow{1}{*}{ 3.06 }  & \multirow{5}{*}{ 16.85 }  & & \\\cline{2-2}
27 & \multirow{1}{*}{ 2.69 }  & & & \\\cline{2-2}
28 & \multirow{1}{*}{ 3.79 }  & & & \\\cline{2-2}
29 & \multirow{1}{*}{ 3.75 }  & & & \\\cline{2-2}
30 & \multirow{1}{*}{ 3.57 }  & & & \\\cline{2-2}\cline{3-3}\cline{4-4}\cline{5-5}
\hline\end{tabular}
\end{center}
\label{tab:min}
\end{table}
\begin{table}[!h]
	\caption{LAD activity along the days of the month.}
	\footnotesize
	\begin{center}
\begin{tabular}{| l || c | c | c | c |}\hline
 & 1 day & 5 & 10 & 15 days \\\hline
1 & \multirow{1}{*}{ 3.29 }  & \multirow{5}{*}{ 15.77 }  & \multirow{10}{*}{ 33.63 }  & \multirow{15}{*}{ 50.50 }  \\\cline{2-2}
2 & \multirow{1}{*}{ 3.38 }  & & & \\\cline{2-2}
3 & \multirow{1}{*}{ 2.85 }  & & & \\\cline{2-2}
4 & \multirow{1}{*}{ 2.94 }  & & & \\\cline{2-2}
5 & \multirow{1}{*}{ 3.31 }  & & & \\\cline{2-2}\cline{3-3}
6 & \multirow{1}{*}{ 3.60 }  & \multirow{5}{*}{ 17.85 }  & & \\\cline{2-2}
7 & \multirow{1}{*}{ 2.68 }  & & & \\\cline{2-2}
8 & \multirow{1}{*}{ 3.78 }  & & & \\\cline{2-2}
9 & \multirow{1}{*}{ 3.88 }  & & & \\\cline{2-2}
10 & \multirow{1}{*}{ 3.91 }  & & & \\\cline{2-2}\cline{3-3}\cline{4-4}
11 & \multirow{1}{*}{ 3.22 }  & \multirow{5}{*}{ 16.87 }  & \multirow{10}{*}{ 33.41 }  & \\\cline{2-2}
12 & \multirow{1}{*}{ 2.79 }  & & & \\\cline{2-2}
13 & \multirow{1}{*}{ 3.50 }  & & & \\\cline{2-2}
14 & \multirow{1}{*}{ 3.95 }  & & & \\\cline{2-2}
15 & \multirow{1}{*}{ 3.40 }  & & & \\\cline{2-2}\cline{3-3}\cline{5-5}
16 & \multirow{1}{*}{ 3.32 }  & \multirow{5}{*}{ 16.54 }  & & \multirow{15}{*}{ 49.50 }  \\\cline{2-2}
17 & \multirow{1}{*}{ 2.95 }  & & & \\\cline{2-2}
18 & \multirow{1}{*}{ 3.50 }  & & & \\\cline{2-2}
19 & \multirow{1}{*}{ 3.69 }  & & & \\\cline{2-2}
20 & \multirow{1}{*}{ 3.07 }  & & & \\\cline{2-2}\cline{3-3}\cline{4-4}
21 & \multirow{1}{*}{ 2.76 }  & \multirow{5}{*}{ 15.71 }  & \multirow{10}{*}{ 32.96 }  & \\\cline{2-2}
22 & \multirow{1}{*}{ 3.35 }  & & & \\\cline{2-2}
23 & \multirow{1}{*}{ 3.32 }  & & & \\\cline{2-2}
24 & \multirow{1}{*}{ 3.15 }  & & & \\\cline{2-2}
25 & \multirow{1}{*}{ 3.13 }  & & & \\\cline{2-2}\cline{3-3}
26 & \multirow{1}{*}{ 3.68 }  & \multirow{5}{*}{ 17.25 }  & & \\\cline{2-2}
27 & \multirow{1}{*}{ 4.02 }  & & & \\\cline{2-2}
28 & \multirow{1}{*}{ 3.49 }  & & & \\\cline{2-2}
29 & \multirow{1}{*}{ 3.34 }  & & & \\\cline{2-2}
30 & \multirow{1}{*}{ 2.72 }  & & & \\\cline{2-2}\cline{3-3}\cline{4-4}\cline{5-5}
\hline\end{tabular}
\end{center}
\label{tab:min}
\end{table}
\begin{table}[!h]
	\caption{MET activity along the days of the month.}
	\footnotesize
	\begin{center}
\begin{tabular}{| l || c | c | c | c |}\hline
 & 1 day & 5 & 10 & 15 days \\\hline
1 & \multirow{1}{*}{ 5.31 }  & \multirow{5}{*}{ 20.06 }  & \multirow{10}{*}{ 37.31 }  & \multirow{15}{*}{ 52.41 }  \\\cline{2-2}
2 & \multirow{1}{*}{ 3.54 }  & & & \\\cline{2-2}
3 & \multirow{1}{*}{ 3.80 }  & & & \\\cline{2-2}
4 & \multirow{1}{*}{ 3.66 }  & & & \\\cline{2-2}
5 & \multirow{1}{*}{ 3.74 }  & & & \\\cline{2-2}\cline{3-3}
6 & \multirow{1}{*}{ 3.84 }  & \multirow{5}{*}{ 17.25 }  & & \\\cline{2-2}
7 & \multirow{1}{*}{ 3.24 }  & & & \\\cline{2-2}
8 & \multirow{1}{*}{ 3.46 }  & & & \\\cline{2-2}
9 & \multirow{1}{*}{ 3.42 }  & & & \\\cline{2-2}
10 & \multirow{1}{*}{ 3.29 }  & & & \\\cline{2-2}\cline{3-3}\cline{4-4}
11 & \multirow{1}{*}{ 3.25 }  & \multirow{5}{*}{ 15.09 }  & \multirow{10}{*}{ 30.63 }  & \\\cline{2-2}
12 & \multirow{1}{*}{ 3.07 }  & & & \\\cline{2-2}
13 & \multirow{1}{*}{ 3.11 }  & & & \\\cline{2-2}
14 & \multirow{1}{*}{ 2.88 }  & & & \\\cline{2-2}
15 & \multirow{1}{*}{ 2.77 }  & & & \\\cline{2-2}\cline{3-3}\cline{5-5}
16 & \multirow{1}{*}{ 3.15 }  & \multirow{5}{*}{ 15.53 }  & & \multirow{15}{*}{ 47.59 }  \\\cline{2-2}
17 & \multirow{1}{*}{ 3.16 }  & & & \\\cline{2-2}
18 & \multirow{1}{*}{ 3.53 }  & & & \\\cline{2-2}
19 & \multirow{1}{*}{ 2.76 }  & & & \\\cline{2-2}
20 & \multirow{1}{*}{ 2.93 }  & & & \\\cline{2-2}\cline{3-3}\cline{4-4}
21 & \multirow{1}{*}{ 3.39 }  & \multirow{5}{*}{ 15.52 }  & \multirow{10}{*}{ 32.06 }  & \\\cline{2-2}
22 & \multirow{1}{*}{ 2.86 }  & & & \\\cline{2-2}
23 & \multirow{1}{*}{ 3.52 }  & & & \\\cline{2-2}
24 & \multirow{1}{*}{ 2.76 }  & & & \\\cline{2-2}
25 & \multirow{1}{*}{ 2.98 }  & & & \\\cline{2-2}\cline{3-3}
26 & \multirow{1}{*}{ 3.09 }  & \multirow{5}{*}{ 16.54 }  & & \\\cline{2-2}
27 & \multirow{1}{*}{ 3.06 }  & & & \\\cline{2-2}
28 & \multirow{1}{*}{ 3.54 }  & & & \\\cline{2-2}
29 & \multirow{1}{*}{ 3.74 }  & & & \\\cline{2-2}
30 & \multirow{1}{*}{ 3.10 }  & & & \\\cline{2-2}\cline{3-3}\cline{4-4}\cline{5-5}
\hline\end{tabular}
\end{center}
\label{tab:min}
\end{table}
\begin{table}[!h]
	\caption{CPP activity along the days of the month.}
	\footnotesize
	\begin{center}
\begin{tabular}{| l || c | c | c | c |}\hline
 & 1 day & 5 & 10 & 15 days \\\hline
1 & \multirow{1}{*}{ 5.00 }  & \multirow{5}{*}{ 18.21 }  & \multirow{10}{*}{ 33.84 }  & \multirow{15}{*}{ 51.29 }  \\\cline{2-2}
2 & \multirow{1}{*}{ 3.02 }  & & & \\\cline{2-2}
3 & \multirow{1}{*}{ 3.65 }  & & & \\\cline{2-2}
4 & \multirow{1}{*}{ 3.18 }  & & & \\\cline{2-2}
5 & \multirow{1}{*}{ 3.37 }  & & & \\\cline{2-2}\cline{3-3}
6 & \multirow{1}{*}{ 3.22 }  & \multirow{5}{*}{ 15.63 }  & & \\\cline{2-2}
7 & \multirow{1}{*}{ 3.54 }  & & & \\\cline{2-2}
8 & \multirow{1}{*}{ 2.59 }  & & & \\\cline{2-2}
9 & \multirow{1}{*}{ 2.87 }  & & & \\\cline{2-2}
10 & \multirow{1}{*}{ 3.41 }  & & & \\\cline{2-2}\cline{3-3}\cline{4-4}
11 & \multirow{1}{*}{ 3.42 }  & \multirow{5}{*}{ 17.46 }  & \multirow{10}{*}{ 33.00 }  & \\\cline{2-2}
12 & \multirow{1}{*}{ 3.50 }  & & & \\\cline{2-2}
13 & \multirow{1}{*}{ 3.30 }  & & & \\\cline{2-2}
14 & \multirow{1}{*}{ 3.48 }  & & & \\\cline{2-2}
15 & \multirow{1}{*}{ 3.76 }  & & & \\\cline{2-2}\cline{3-3}\cline{5-5}
16 & \multirow{1}{*}{ 3.37 }  & \multirow{5}{*}{ 15.54 }  & & \multirow{15}{*}{ 48.71 }  \\\cline{2-2}
17 & \multirow{1}{*}{ 3.45 }  & & & \\\cline{2-2}
18 & \multirow{1}{*}{ 3.01 }  & & & \\\cline{2-2}
19 & \multirow{1}{*}{ 2.97 }  & & & \\\cline{2-2}
20 & \multirow{1}{*}{ 2.75 }  & & & \\\cline{2-2}\cline{3-3}\cline{4-4}
21 & \multirow{1}{*}{ 3.26 }  & \multirow{5}{*}{ 16.97 }  & \multirow{10}{*}{ 33.17 }  & \\\cline{2-2}
22 & \multirow{1}{*}{ 3.70 }  & & & \\\cline{2-2}
23 & \multirow{1}{*}{ 3.20 }  & & & \\\cline{2-2}
24 & \multirow{1}{*}{ 3.22 }  & & & \\\cline{2-2}
25 & \multirow{1}{*}{ 3.59 }  & & & \\\cline{2-2}\cline{3-3}
26 & \multirow{1}{*}{ 3.51 }  & \multirow{5}{*}{ 16.20 }  & & \\\cline{2-2}
27 & \multirow{1}{*}{ 3.12 }  & & & \\\cline{2-2}
28 & \multirow{1}{*}{ 3.26 }  & & & \\\cline{2-2}
29 & \multirow{1}{*}{ 3.70 }  & & & \\\cline{2-2}
30 & \multirow{1}{*}{ 2.62 }  & & & \\\cline{2-2}\cline{3-3}\cline{4-4}\cline{5-5}
\hline\end{tabular}
\end{center}
\label{tab:min}
\end{table}

\FloatBarrier
\subsubsection{Histograms of activity along months of the year}\label{si:months}
	Activity is concentrated in Jun-Aug and/or in Dec-Mar.
	These observations mostly fit academic calendars, vacations and end-of-year holidays.
\begin{table}[!h]
	\caption{LAU activity along the months of the year.}
	\footnotesize
	\begin{center}
\begin{tabular}{| l || c | c | c | c | c |}\hline
 & m. & b. & t. & q. & s. \\\hline
Jan & \multirow{1}{*}{ 10.22 }  & \multirow{2}{*}{ 19.56 }  & \multirow{3}{*}{ 28.24 }  & \multirow{4}{*}{ 35.09 }  & \multirow{6}{*}{ 49.16 }  \\\cline{2-2}
Fev & \multirow{1}{*}{ 9.34 }  & & & & \\\cline{2-2}\cline{3-3}
Mar & \multirow{1}{*}{ 8.67 }  & \multirow{2}{*}{ 15.53 }  & & & \\\cline{2-2}\cline{4-4}
Apr & \multirow{1}{*}{ 6.86 }  & & \multirow{3}{*}{ 20.93 }  & & \\\cline{2-2}\cline{3-3}\cline{5-5}
Mai & \multirow{1}{*}{ 7.28 }  & \multirow{2}{*}{ 14.07 }  & & \multirow{4}{*}{ 30.36 }  & \\\cline{2-2}
Jun & \multirow{1}{*}{ 6.80 }  & & & & \\\cline{2-2}\cline{3-3}\cline{4-4}\cline{6-6}
Jul & \multirow{1}{*}{ 8.97 }  & \multirow{2}{*}{ 16.29 }  & \multirow{3}{*}{ 24.47 }  & & \multirow{6}{*}{ 50.84 }  \\\cline{2-2}
Ago & \multirow{1}{*}{ 7.32 }  & & & & \\\cline{2-2}\cline{3-3}\cline{5-5}
Set & \multirow{1}{*}{ 8.18 }  & \multirow{2}{*}{ 16.25 }  & & \multirow{4}{*}{ 34.55 }  & \\\cline{2-2}\cline{4-4}
Out & \multirow{1}{*}{ 8.06 }  & & \multirow{3}{*}{ 26.36 }  & & \\\cline{2-2}\cline{3-3}
Nov & \multirow{1}{*}{ 7.64 }  & \multirow{2}{*}{ 18.30 }  & & & \\\cline{2-2}
Dez & \multirow{1}{*}{ 10.66 }  & & & & \\\cline{2-2}\cline{3-3}\cline{4-4}\cline{5-5}\cline{6-6}
\hline\end{tabular}
\end{center}
\label{tab:min2}
\end{table}
\begin{table}[!h]
	\caption{LAD activity along the months of the year.}
	\footnotesize
	\begin{center}
\begin{tabular}{| l || c | c | c | c | c |}\hline
 & m. & b. & t. & q. & s. \\\hline
Jan & \multirow{1}{*}{ 11.24 }  & \multirow{2}{*}{ 18.51 }  & \multirow{3}{*}{ 26.46 }  & \multirow{4}{*}{ 36.07 }  & \multirow{6}{*}{ 57.96 }  \\\cline{2-2}
Fev & \multirow{1}{*}{ 7.26 }  & & & & \\\cline{2-2}\cline{3-3}
Mar & \multirow{1}{*}{ 7.95 }  & \multirow{2}{*}{ 17.56 }  & & & \\\cline{2-2}\cline{4-4}
Apr & \multirow{1}{*}{ 9.61 }  & & \multirow{3}{*}{ 31.50 }  & & \\\cline{2-2}\cline{3-3}\cline{5-5}
Mai & \multirow{1}{*}{ 8.94 }  & \multirow{2}{*}{ 21.89 }  & & \multirow{4}{*}{ 37.56 }  & \\\cline{2-2}
Jun & \multirow{1}{*}{ 12.95 }  & & & & \\\cline{2-2}\cline{3-3}\cline{4-4}\cline{6-6}
Jul & \multirow{1}{*}{ 9.03 }  & \multirow{2}{*}{ 15.67 }  & \multirow{3}{*}{ 22.30 }  & & \multirow{6}{*}{ 42.04 }  \\\cline{2-2}
Ago & \multirow{1}{*}{ 6.64 }  & & & & \\\cline{2-2}\cline{3-3}\cline{5-5}
Set & \multirow{1}{*}{ 6.63 }  & \multirow{2}{*}{ 12.38 }  & & \multirow{4}{*}{ 26.37 }  & \\\cline{2-2}\cline{4-4}
Out & \multirow{1}{*}{ 5.75 }  & & \multirow{3}{*}{ 19.74 }  & & \\\cline{2-2}\cline{3-3}
Nov & \multirow{1}{*}{ 7.61 }  & \multirow{2}{*}{ 13.99 }  & & & \\\cline{2-2}
Dez & \multirow{1}{*}{ 6.38 }  & & & & \\\cline{2-2}\cline{3-3}\cline{4-4}\cline{5-5}\cline{6-6}
\hline\end{tabular}
\end{center}
\label{tab:min2}
\end{table}
\begin{table}[!h]
	\caption{MET activity along the months of the year.}
	\footnotesize
	\begin{center}
\begin{tabular}{| l || c | c | c | c | c |}\hline
 & m. & b. & t. & q. & s. \\\hline
Jan & \multirow{1}{*}{ 4.87 }  & \multirow{2}{*}{ 11.00 }  & \multirow{3}{*}{ 16.89 }  & \multirow{4}{*}{ 23.30 }  & \multirow{6}{*}{ 47.70 }  \\\cline{2-2}
Fev & \multirow{1}{*}{ 6.13 }  & & & & \\\cline{2-2}\cline{3-3}
Mar & \multirow{1}{*}{ 5.89 }  & \multirow{2}{*}{ 12.30 }  & & & \\\cline{2-2}\cline{4-4}
Apr & \multirow{1}{*}{ 6.41 }  & & \multirow{3}{*}{ 30.81 }  & & \\\cline{2-2}\cline{3-3}\cline{5-5}
Mai & \multirow{1}{*}{ 10.45 }  & \multirow{2}{*}{ 24.40 }  & & \multirow{4}{*}{ 47.87 }  & \\\cline{2-2}
Jun & \multirow{1}{*}{ 13.95 }  & & & & \\\cline{2-2}\cline{3-3}\cline{4-4}\cline{6-6}
Jul & \multirow{1}{*}{ 13.24 }  & \multirow{2}{*}{ 23.47 }  & \multirow{3}{*}{ 31.21 }  & & \multirow{6}{*}{ 52.30 }  \\\cline{2-2}
Ago & \multirow{1}{*}{ 10.22 }  & & & & \\\cline{2-2}\cline{3-3}\cline{5-5}
Set & \multirow{1}{*}{ 7.75 }  & \multirow{2}{*}{ 16.79 }  & & \multirow{4}{*}{ 28.83 }  & \\\cline{2-2}\cline{4-4}
Out & \multirow{1}{*}{ 9.04 }  & & \multirow{3}{*}{ 21.09 }  & & \\\cline{2-2}\cline{3-3}
Nov & \multirow{1}{*}{ 7.45 }  & \multirow{2}{*}{ 12.05 }  & & & \\\cline{2-2}
Dez & \multirow{1}{*}{ 4.59 }  & & & & \\\cline{2-2}\cline{3-3}\cline{4-4}\cline{5-5}\cline{6-6}
\hline\end{tabular}
\end{center}
\label{tab:min2}
\end{table}

\begin{table}[!h]
	\caption{CPP activity along the months of the year.}
	\footnotesize
	\begin{center}
\begin{tabular}{| l || c | c | c | c | c |}\hline
 & m. & b. & t. & q. & s. \\\hline
Jan & \multirow{1}{*}{ 8.70 }  & \multirow{2}{*}{ 17.00 }  & \multirow{3}{*}{ 27.23 }  & \multirow{4}{*}{ 36.49 }  & \multirow{6}{*}{ 54.27 }  \\\cline{2-2}
Fev & \multirow{1}{*}{ 8.29 }  & & & & \\\cline{2-2}\cline{3-3}
Mar & \multirow{1}{*}{ 10.23 }  & \multirow{2}{*}{ 19.49 }  & & & \\\cline{2-2}\cline{4-4}
Apr & \multirow{1}{*}{ 9.26 }  & & \multirow{3}{*}{ 27.03 }  & & \\\cline{2-2}\cline{3-3}\cline{5-5}
Mai & \multirow{1}{*}{ 9.41 }  & \multirow{2}{*}{ 17.78 }  & & \multirow{4}{*}{ 33.46 }  & \\\cline{2-2}
Jun & \multirow{1}{*}{ 8.37 }  & & & & \\\cline{2-2}\cline{3-3}\cline{4-4}\cline{6-6}
Jul & \multirow{1}{*}{ 8.70 }  & \multirow{2}{*}{ 15.68 }  & \multirow{3}{*}{ 22.94 }  & & \multirow{6}{*}{ 45.73 }  \\\cline{2-2}
Ago & \multirow{1}{*}{ 6.98 }  & & & & \\\cline{2-2}\cline{3-3}\cline{5-5}
Set & \multirow{1}{*}{ 7.26 }  & \multirow{2}{*}{ 15.36 }  & & \multirow{4}{*}{ 30.06 }  & \\\cline{2-2}\cline{4-4}
Out & \multirow{1}{*}{ 8.10 }  & & \multirow{3}{*}{ 22.80 }  & & \\\cline{2-2}\cline{3-3}
Nov & \multirow{1}{*}{ 7.89 }  & \multirow{2}{*}{ 14.69 }  & & & \\\cline{2-2}
Dez & \multirow{1}{*}{ 6.81 }  & & & & \\\cline{2-2}\cline{3-3}\cline{4-4}\cline{5-5}\cline{6-6}
\hline\end{tabular}
\end{center}
\label{tab:min2}
\end{table}


\FloatBarrier
\section{PCA of measures along the timeline}\label{si:pcat}
Loadings for topological metrics in principal components for LAD, LAU, MET, CPP, lists with the window size of $ws=1000$ messages 20 disjoint positioning.
Further details are given in Sections~\ref{measures} and~\ref{prevalence} of the main document~\cite{tpaper}.



\subsection{Betweenness, clustering and degree}
This simplest PCA is characterized by the coupling of
the centrality measures of degree $k$ and betweenness $bt$
in the first component.
Second component is mostly the clustering coefficient $cc$.
These three measures contribute almost equally to the dispersion of the system: first component holds two thirds of the dispersion and is resultant of two measures while the second component holds one third of the dispersion and results from one measure.

\begin{table}[!h]
	\caption{LAU principal components formation and concentration of dispersion.}
	\footnotesize
	\begin{center}
\begin{tabular}{| l || c | c | c | c | c | c |}\cline{2-7}
\multicolumn{1}{c|}{} & \multicolumn{2}{c|}{PC1}          & \multicolumn{2}{c|}{PC2} & \multicolumn{2}{c|}{PC3}  \\\cline{2-7}\multicolumn{1}{c|}{} & $\mu$            & $\sigma$ & $\mu$         & $\sigma$ & $\mu$ & $\sigma$  \\\hline
$cc$ & 6.03  & 3.73  & 87.60  & 5.25  & 4.52  & 0.93 \\
$k$ & 47.13  & 1.76  & 3.01  & 1.98  & 47.90  & 0.38 \\
$bt$ & 46.84  & 1.97  & 9.39  & 4.31  & 47.58  & 0.57 \\\hline\hline
$\lambda$ & 64.99  & 0.60  & 33.08  & 0.41  & 1.93  & 0.36 \\
\hline\end{tabular}
\end{center}

\label{tab:pcain}
\end{table}

\begin{table}[!h]
	\caption{LAD principal components formation and concentration of dispersion.}
	\footnotesize
	\begin{center}
\begin{tabular}{| l || c | c | c | c | c | c |}\cline{2-7}
\multicolumn{1}{c|}{} & \multicolumn{2}{c|}{PC1}          & \multicolumn{2}{c|}{PC2} & \multicolumn{2}{c|}{PC3}  \\\cline{2-7}\multicolumn{1}{c|}{} & $\mu$            & $\sigma$ & $\mu$         & $\sigma$ & $\mu$ & $\sigma$  \\\hline
$cc$ & 6.42  & 4.05  & 86.60  & 5.50  & 5.19  & 1.45 \\
$k$ & 46.98  & 1.86  & 2.95  & 1.65  & 47.61  & 0.57 \\
$bt$ & 46.59  & 2.18  & 10.45  & 4.72  & 47.20  & 0.90 \\\hline\hline
$\lambda$ & 64.96  & 0.71  & 33.08  & 0.41  & 1.96  & 0.52 \\
\hline\end{tabular}
\end{center}

\label{tab:pcain}
\end{table}

\begin{table}[!h]
	\caption{MET principal components formation and concentration of dispersion.}
	\footnotesize
	\begin{center}
\begin{tabular}{| l || c | c | c | c | c | c |}\cline{2-7}
\multicolumn{1}{c|}{} & \multicolumn{2}{c|}{PC1}          & \multicolumn{2}{c|}{PC2} & \multicolumn{2}{c|}{PC3}  \\\cline{2-7}\multicolumn{1}{c|}{} & $\mu$            & $\sigma$ & $\mu$         & $\sigma$ & $\mu$ & $\sigma$  \\\hline
$cc$ & 5.82  & 3.76  & 87.26  & 5.12  & 4.93  & 1.19 \\
$k$ & 47.18  & 1.82  & 4.35  & 4.01  & 47.63  & 0.57 \\
$bt$ & 47.01  & 1.96  & 8.40  & 4.22  & 47.44  & 0.67 \\\hline\hline
$\lambda$ & 64.94  & 0.76  & 33.13  & 0.45  & 1.93  & 0.62 \\
\hline\end{tabular}
\end{center}

\label{tab:pcain}
\end{table}

\begin{table}[!h]
	\caption{CPP principal components formation and concentration of dispersion.}
	\footnotesize
	\begin{center}
\begin{tabular}{| l || c | c | c | c | c | c |}\cline{2-7}
\multicolumn{1}{c|}{} & \multicolumn{2}{c|}{PC1}          & \multicolumn{2}{c|}{PC2} & \multicolumn{2}{c|}{PC3}  \\\cline{2-7}\multicolumn{1}{c|}{} & $\mu$            & $\sigma$ & $\mu$         & $\sigma$ & $\mu$ & $\sigma$  \\\hline
$cc$ & 3.61  & 2.13  & 91.86  & 3.24  & 3.59  & 0.98 \\
$k$ & 48.24  & 0.99  & 2.96  & 2.25  & 48.25  & 0.43 \\
$bt$ & 48.15  & 1.14  & 5.18  & 3.89  & 48.16  & 0.56 \\\hline\hline
$\lambda$ & 65.24  & 0.51  & 33.30  & 0.17  & 1.46  & 0.49 \\
\hline\end{tabular}
\end{center}

\label{tab:pcain}
\end{table}

\FloatBarrier
\subsection{Betweenness, clustering, degrees and strengths}
In this extension of the previous plot, the set of centrality metrics are extended to include strength $s$ and the in- and out- degrees ($k^{in}$, $k^{out}$) and strengths ($s^{in}$, $s^{out}$).
First component holds an average of the centrality metrics ($k$, $k^{in}$, $k^{out}$, $s$, $s^{in}$, $s^{out}$) while
second component is mostly clustering coefficient $cc$.
All metrics contribute the about same for total dispersion
($\approx \frac{94}{8}=11.75\%$).



\begin{table}[!h]
	\caption{LAU principal components formation and concentration of dispersion.}
	\footnotesize
	\begin{center}
\begin{tabular}{| l || c | c | c | c | c | c |}\cline{2-7}
\multicolumn{1}{c|}{} & \multicolumn{2}{c|}{PC1}          & \multicolumn{2}{c|}{PC2} & \multicolumn{2}{c|}{PC3}  \\\cline{2-7}\multicolumn{1}{c|}{} & $\mu$            & $\sigma$ & $\mu$         & $\sigma$ & $\mu$ & $\sigma$  \\\hline
$cc$ & 1.59  & 0.81  & 80.37  & 5.18  & 3.09  & 1.89 \\\hline
$s$ & 14.40  & 0.15  & 0.81  & 0.68  & 4.75  & 4.43 \\
$s^{in}$ & 14.00  & 0.14  & 2.32  & 1.49  & 18.98  & 4.93 \\
$s^{out}$ & 13.96  & 0.14  & 2.72  & 1.44  & 18.25  & 6.36 \\
$k$ & 14.49  & 0.15  & 0.54  & 0.35  & 1.37  & 0.98 \\
$k^{in}$ & 14.01  & 0.13  & 2.72  & 1.35  & 18.69  & 5.01 \\
$k^{out}$ & 13.85  & 0.13  & 2.37  & 1.73  & 22.63  & 3.79 \\
$bt$ & 13.69  & 0.22  & 8.16  & 1.62  & 12.23  & 8.33 \\\hline\hline
$\lambda$ & 81.87  & 0.88  & 12.48  & 0.15  & 3.33  & 0.70 \\
\hline\end{tabular}
\end{center}

\label{tab:pcain}
\end{table}
\begin{table}[!h]
	\caption{LAD principal components formation and concentration of dispersion.}
	\footnotesize
	\begin{center}
\begin{tabular}{| l || c | c | c | c | c | c |}\cline{2-7}
\multicolumn{1}{c|}{} & \multicolumn{2}{c|}{PC1}          & \multicolumn{2}{c|}{PC2} & \multicolumn{2}{c|}{PC3}  \\\cline{2-7}\multicolumn{1}{c|}{} & $\mu$            & $\sigma$ & $\mu$         & $\sigma$ & $\mu$ & $\sigma$  \\\hline
$cc$ & 1.83  & 1.11  & 80.38  & 11.45  & 3.78  & 4.43 \\\hline
$s$ & 14.25  & 0.17  & 1.34  & 1.81  & 9.88  & 5.76 \\
$s^{in}$ & 13.99  & 0.19  & 2.06  & 1.70  & 17.62  & 6.15 \\
$s^{out}$ & 14.03  & 0.22  & 1.81  & 1.98  & 15.44  & 6.68 \\
$k$ & 14.38  & 0.13  & 0.95  & 1.64  & 3.45  & 3.15 \\
$k^{in}$ & 14.05  & 0.14  & 2.26  & 1.66  & 13.44  & 7.26 \\
$k^{out}$ & 13.96  & 0.15  & 1.72  & 1.53  & 16.14  & 6.37 \\
$bt$ & 13.51  & 0.35  & 9.48  & 2.86  & 20.26  & 9.87 \\\hline\hline
$\lambda$ & 82.32  & 1.61  & 12.52  & 0.26  & 2.97  & 1.21 \\
\hline\end{tabular}
\end{center}

\label{tab:pcain}
\end{table}
\begin{table}[!h]
	\caption{MET principal components formation and concentration of dispersion.}
	\footnotesize
	\begin{center}
\begin{tabular}{| l || c | c | c | c | c | c |}\cline{2-7}
\multicolumn{1}{c|}{} & \multicolumn{2}{c|}{PC1}          & \multicolumn{2}{c|}{PC2} & \multicolumn{2}{c|}{PC3}  \\\cline{2-7}\multicolumn{1}{c|}{} & $\mu$            & $\sigma$ & $\mu$         & $\sigma$ & $\mu$ & $\sigma$  \\\hline
$cc$ & 1.16  & 0.76  & 81.72  & 3.00  & 1.61  & 1.78 \\\hline
$s$ & 14.32  & 0.16  & 1.76  & 1.12  & 11.39  & 5.50 \\
$s^{in}$ & 14.17  & 0.11  & 2.29  & 1.29  & 14.46  & 3.72 \\
$s^{out}$ & 14.09  & 0.17  & 1.72  & 1.18  & 17.54  & 5.37 \\
$k$ & 14.39  & 0.16  & 1.73  & 0.63  & 4.76  & 2.82 \\
$k^{in}$ & 14.12  & 0.13  & 1.02  & 0.71  & 11.69  & 6.93 \\
$k^{out}$ & 14.06  & 0.13  & 3.11  & 1.58  & 12.18  & 9.24 \\
$bt$ & 13.69  & 0.26  & 6.64  & 2.01  & 26.37  & 12.37 \\\hline\hline
$\lambda$ & 83.41  & 1.53  & 12.53  & 0.11  & 2.34  & 1.16 \\
\hline\end{tabular}
\end{center}

\label{tab:pcain}
\end{table}
\begin{table}[!h]
	\caption{CPP principal components formation and concentration of dispersion.}
	\footnotesize
	\begin{center}
\begin{tabular}{| l || c | c | c | c | c | c |}\cline{2-7}
\multicolumn{1}{c|}{} & \multicolumn{2}{c|}{PC1}          & \multicolumn{2}{c|}{PC2} & \multicolumn{2}{c|}{PC3}  \\\cline{2-7}\multicolumn{1}{c|}{} & $\mu$            & $\sigma$ & $\mu$         & $\sigma$ & $\mu$ & $\sigma$  \\\hline
$cc$ & 0.84  & 0.61  & 80.59  & 6.89  & 2.30  & 2.19 \\\hline
$s$ & 14.28  & 0.07  & 0.97  & 1.03  & 15.89  & 1.15 \\
$s^{in}$ & 14.18  & 0.12  & 2.89  & 1.71  & 13.50  & 5.19 \\
$s^{out}$ & 14.07  & 0.23  & 2.83  & 1.63  & 18.80  & 4.94 \\
$k$ & 14.42  & 0.08  & 0.78  & 0.67  & 7.48  & 2.71 \\
$k^{in}$ & 14.29  & 0.10  & 2.36  & 1.41  & 7.21  & 4.49 \\
$k^{out}$ & 14.16  & 0.12  & 3.62  & 1.83  & 8.79  & 4.58 \\
$bt$ & 13.76  & 0.22  & 5.96  & 1.88  & 26.03  & 7.94 \\\hline\hline
$\lambda$ & 83.32  & 1.42  & 12.60  & 0.08  & 2.61  & 1.15 \\
\hline\end{tabular}
\end{center}

\label{tab:pcain}
\end{table}


\FloatBarrier
\subsection{Betweenness, clustering, degrees, strengths and symmetry measures}

Loadings for 14 topological metrics in the first three principal components are given for LAD, LAU, MET, CPP, list..
The clustering coefficient $cc$ appears as the first metric in the tables, followed by 7 centrality metrics ($k$, $k^{in}$, $k^{out}$, $s$, $s^{in}$, $s^{ou}$, $bt$) and 6 symmetry-related metrics ($asy$, $\mu^{asy}$, $\sigma^{asy}$, $dis$, $\mu^{dis}$ and $\sigma^{dis}$).
The centrality metrics are most important for the first principal component, while the second component is predominantly the result of symmetry metrics.
The clustering coefficient is only relevant for the third principal component, coupled with standard deviations of asymmetry $\sigma^{asy}$ and disequilibrium $\sigma^{dis}$.
The three components sum up to $\approx90\%$ of the variance.
In the PCA measures of the CPP list, the last table of these PCA-related tables, this general pattern is depicted in {\bf boldface}.
Further details are given in Sections~\ref{measures} and~\ref{prevalence} of the main document~\cite{tpaper}.

\begin{table}[!h]
	\caption{LAU principal components formation and concentration of dispersion.}
	\footnotesize
	\begin{center}
\begin{tabular}{| l || c | c | c | c | c | c |}\cline{2-7}
\multicolumn{1}{c|}{} & \multicolumn{2}{c|}{PC1}          & \multicolumn{2}{c|}{PC2} & \multicolumn{2}{c|}{PC3}  \\\cline{2-7}\multicolumn{1}{c|}{} & $\mu$            & $\sigma$ & $\mu$         & $\sigma$ & $\mu$ & $\sigma$  \\\hline
$cc$ & 1.64  & 0.77  & 2.42  & 1.71  & 19.20  & 3.96 \\\hline
$s$ & 12.80  & 0.46  & 0.89  & 0.82  & 2.53  & 0.63 \\
$s^{in}$ & 12.47  & 0.42  & 2.30  & 0.97  & 2.29  & 0.81 \\
$s^{out}$ & 12.37  & 0.46  & 2.89  & 1.24  & 2.64  & 0.58 \\
$k$ & 12.93  & 0.44  & 0.82  & 0.73  & 1.32  & 0.45 \\
$k^{in}$ & 12.54  & 0.37  & 2.88  & 1.13  & 1.02  & 0.56 \\
$k^{out}$ & 12.32  & 0.46  & 3.82  & 1.14  & 1.57  & 0.68 \\
$bt$ & 12.19  & 0.46  & 1.06  & 0.62  & 2.64  & 0.89 \\\hline
$asy$ & 0.93  & 0.81  & 20.38  & 0.82  & 1.66  & 1.09 \\
$\mu^{asy}$ & 0.96  & 0.83  & 20.26  & 0.82  & 1.66  & 1.04 \\
$\sigma^{asy}$ & 6.18  & 0.71  & 1.24  & 0.92  & 27.98  & 1.74 \\
$dis$ & 0.90  & 0.79  & 20.36  & 0.82  & 1.54  & 1.07 \\
$\mu^{dis}$ & 0.92  & 0.61  & 19.02  & 0.84  & 1.45  & 1.12 \\
$\sigma^{dis}$ & 0.86  & 0.51  & 1.64  & 1.10  & 32.51  & 1.90 \\\hline\hline
$\lambda$ & 48.41  & 0.52  & 27.95  & 0.36  & 12.81  & 0.79 \\
\hline\end{tabular}
\end{center}

\label{tab:pcain}
\end{table}
\begin{table}[!h]
	\caption{LAD principal components formation and concentration of dispersion.}
	\footnotesize
	\begin{center}
\begin{tabular}{| l || c | c | c | c | c | c |}\cline{2-7}
\multicolumn{1}{c|}{} & \multicolumn{2}{c|}{PC1}          & \multicolumn{2}{c|}{PC2} & \multicolumn{2}{c|}{PC3}  \\\cline{2-7}\multicolumn{1}{c|}{} & $\mu$            & $\sigma$ & $\mu$         & $\sigma$ & $\mu$ & $\sigma$  \\\hline
$cc$ & 1.96  & 0.95  & 3.07  & 1.46  & 17.94  & 5.38 \\\hline
$s$ & 12.34  & 0.57  & 1.72  & 0.99  & 2.43  & 0.93 \\
$s^{in}$ & 12.06  & 0.64  & 3.18  & 0.98  & 1.98  & 1.09 \\
$s^{out}$ & 12.22  & 0.48  & 1.14  & 0.78  & 2.83  & 0.79 \\
$k$ & 12.54  & 0.56  & 1.43  & 0.87  & 0.92  & 0.44 \\
$k^{in}$ & 12.15  & 0.61  & 3.81  & 0.79  & 0.61  & 0.42 \\
$k^{out}$ & 12.27  & 0.45  & 1.51  & 1.08  & 1.56  & 0.39 \\
$bt$ & 11.73  & 0.64  & 1.80  & 0.88  & 2.28  & 1.00 \\\hline
$asy$ & 1.51  & 0.97  & 19.66  & 1.63  & 3.02  & 1.66 \\
$\mu^{asy}$ & 1.41  & 0.99  & 19.53  & 1.62  & 3.00  & 1.69 \\
$\sigma^{asy}$ & 5.62  & 0.68  & 2.01  & 1.23  & 27.46  & 3.31 \\
$dis$ & 1.58  & 0.98  & 19.57  & 1.65  & 3.21  & 1.71 \\
$\mu^{dis}$ & 1.84  & 1.00  & 18.62  & 1.52  & 2.08  & 1.13 \\
$\sigma^{dis}$ & 0.77  & 0.59  & 2.94  & 1.60  & 30.68  & 3.34 \\\hline\hline
$\lambda$ & 48.65  & 1.03  & 27.84  & 0.31  & 13.00  & 0.77 \\
\hline\end{tabular}
\end{center}

\label{tab:pcain}
\end{table}
\begin{table}[!h]
	\caption{MET principal components formation and concentration of dispersion.}
	\footnotesize
	\begin{center}
\begin{tabular}{| l || c | c | c | c | c | c |}\cline{2-7}
\multicolumn{1}{c|}{} & \multicolumn{2}{c|}{PC1}          & \multicolumn{2}{c|}{PC2} & \multicolumn{2}{c|}{PC3}  \\\cline{2-7}\multicolumn{1}{c|}{} & $\mu$            & $\sigma$ & $\mu$         & $\sigma$ & $\mu$ & $\sigma$  \\\hline
$cc$ & 1.18  & 0.71  & 3.00  & 2.35  & 22.39  & 2.71 \\\hline
$s$ & 12.34  & 0.66  & 1.74  & 1.17  & 1.55  & 0.75 \\
$s^{in}$ & 12.25  & 0.62  & 1.74  & 0.96  & 1.45  & 0.77 \\
$s^{out}$ & 12.11  & 0.72  & 2.42  & 1.35  & 1.78  & 0.78 \\
$k$ & 12.48  & 0.63  & 1.46  & 0.91  & 0.54  & 0.48 \\
$k^{in}$ & 12.32  & 0.56  & 1.54  & 1.22  & 0.65  & 0.62 \\
$k^{out}$ & 12.12  & 0.67  & 3.10  & 1.15  & 0.87  & 0.74 \\
$bt$ & 11.85  & 0.62  & 1.46  & 0.87  & 1.16  & 0.70 \\\hline
$asy$ & 1.79  & 1.22  & 19.35  & 2.15  & 3.29  & 2.15 \\
$\mu^{asy}$ & 1.84  & 1.22  & 19.17  & 2.16  & 3.31  & 2.23 \\
$\sigma^{asy}$ & 4.17  & 0.79  & 3.91  & 2.35  & 27.79  & 3.96 \\
$dis$ & 1.78  & 1.18  & 19.26  & 2.15  & 3.38  & 2.29 \\
$\mu^{dis}$ & 1.53  & 1.10  & 18.23  & 2.12  & 3.32  & 1.71 \\
$\sigma^{dis}$ & 2.23  & 0.93  & 3.61  & 2.38  & 28.54  & 3.23 \\\hline\hline
$\lambda$ & 49.05  & 1.01  & 27.79  & 0.30  & 13.30  & 1.35 \\
\hline\end{tabular}
\end{center}

\label{tab:pcain}
\end{table}
\begin{table}[!h]
	\caption{CPP principal components formation and concentration of dispersion.}
	\footnotesize
	\begin{center}
\begin{tabular}{| l || c | c | c | c | c | c |}\cline{2-7}
\multicolumn{1}{c|}{} & \multicolumn{2}{c|}{PC1}          & \multicolumn{2}{c|}{PC2} & \multicolumn{2}{c|}{PC3}  \\\cline{2-7}\multicolumn{1}{c|}{} & $\mu$            & $\sigma$ & $\mu$         & $\sigma$ & $\mu$ & $\sigma$  \\\hline
$cc$ &                     0.89  & 0.59  & 1.93  & 1.33  & {\bf 21.22}  & 2.97 \\\hline
$s$ &              {\bf 11.71}  & 0.57  & 2.97  & 0.82  & 2.45  & 0.72 \\
$s^{in}$ &         {\bf 11.68}  & 0.58  & 2.37  & 0.91  & 3.08  & 0.78 \\
$s^{out}$ &        {\bf 11.49}  & 0.61  & 3.63  & 0.79  & 1.61  & 0.88 \\
$k$ &              {\bf 11.93}  & 0.54  & 2.58  & 0.70  & 0.52  & 0.44 \\
$k^{in}$ &         {\bf 11.93}  & 0.52  & 1.19  & 0.88  & 1.41  & 0.71 \\
$k^{out}$ &        {\bf 11.57}  & 0.61  & 4.34  & 0.70  & 0.98  & 0.66 \\
$bt$ &             {\bf 11.37}  & 0.55  & 2.44  & 0.84  & 1.37  & 0.77 \\\hline
$asy$ &                    3.14  & 0.98  & {\bf 18.52}  & 1.97  & 2.46  & 1.69 \\
$\mu^{asy}$              & 3.32  & 0.99  & {\bf 18.23}  & 2.01  & 2.80  & 1.82 \\
$\sigma^{asy}$           & 4.91  & 0.59  & 2.44  & 1.47  & {\bf 26.84}  & 3.06 \\
$dis$                    & 2.94  & 0.88  & {\bf 18.50}  & 1.92  & 3.06  & 1.98 \\
$\mu^{dis}$              & 2.55  & 0.89  & {\bf 18.12}  & 1.85  & 1.57  & 1.32 \\
	$\sigma^{dis}$           & 0.57  & 0.33  & 2.74  & 1.63  & {\bf 30.61}  & 2.66 \\\hline\hline
$\lambda$                & 49.56 & 1.16  & 27.14  & 0.54  & 13.25  & 0.95 \\
\hline\end{tabular}
\end{center}

\label{tab:pcain}
\end{table}


\FloatBarrier
\section{Fraction of participants in each Erd\"os Sector along the timeline}\label{si:frac}
In this section, the figures present timelines
with the 
fractions of participants in each Erd\"os sector with respect to each criterion defined in Section~\ref{sectioning} of the main document~\cite{tpaper}. Step sizes of 50, 100, 250, 500, 1000 and 5000 are shown below, first for CPP, then for  LAD list.

Each step size takes two pages of plot.
On the first page, the criterion is based on each centrality metric observed separately: total, in and out degrees ($k$, $k^{in}$, $k^{out}$) and strengths ($s$, $s^{in}$, $s^{out}$).
In the first six plots of every page,
the colors have meanings as follows: {\bf \color{red} red for hubs}, {\color{Green}\bf green for the intermediary} and {\bf\color{blue} blue for the periphery}.

On the last plot of the first page, red is the center (maximum distance to another vertex is equal to radius, i.e. smallest geodesic), blue is periphery (maximum distance equals to diameter, i.e. greatest geodesic) of the greatest component. On the same graph, green represents the disconnected vertices.

On the second page we show the fractions of participants with respect to each compound criterion for the Erd\"os sectioning.
In the first plot, the {\bf \color{black} fraction of vertices with unique classification is in black}: $\frac{\text{number of nodes uniquely classified}}{\text{number of nodes}}$. On the second plot, {\bf black represents the fraction of classifications over the number of vertices}: $\frac{\text{number of classifications} - \text{number of nodes}}{\text{number of nodes}}$.



\subsection{CPP list}
\begin{figure*}
   \centering
        \includegraphics[width=\textwidth]{evoTimelines/evoTimelineCPP-50/CPP-W50-S200}
\end{figure*}
\begin{figure*}
   \centering
        \includegraphics[width=\textwidth]{evoTimelines/evoTimelineCPP-50/CPP-W50-S200_}
\end{figure*}

\begin{figure*}
   \centering
        \includegraphics[width=\textwidth]{evoTimelines/evoTimelineCPP-100/CPP-W100-S200}
\end{figure*}
\begin{figure*}
   \centering
        \includegraphics[width=\textwidth]{evoTimelines/evoTimelineCPP-100/CPP-W100-S200_}
\end{figure*}

\begin{figure*}
   \centering
        \includegraphics[width=\textwidth]{evoTimelines/evoTimelineCPP-250/CPP-W250-S250}
\end{figure*}
\begin{figure*}
   \centering
        \includegraphics[width=\textwidth]{evoTimelines/evoTimelineCPP-250/CPP-W250-S250_}
\end{figure*}

\begin{figure*}
   \centering
        \includegraphics[width=\textwidth]{evoTimelines/evoTimelineCPP-500/CPP-W500-S500}
\end{figure*}
\begin{figure*}
   \centering
        \includegraphics[width=\textwidth]{evoTimelines/evoTimelineCPP-500/CPP-W500-S500_}
\end{figure*}

\begin{figure*}
   \centering
        \includegraphics[width=\textwidth]{evoTimelines/evoTimelineCPP-1000/CPP-W1000-S1000}
\end{figure*}
\begin{figure*}
   \centering
        \includegraphics[width=\textwidth]{evoTimelines/evoTimelineCPP-1000/CPP-W1000-S1000_}
\end{figure*}

\begin{figure*}
   \centering
        \includegraphics[width=\textwidth]{evoTimelines/evoTimelineCPP-3300/CPP-W3300-S3300}
\end{figure*}
\begin{figure*}
   \centering
   \includegraphics[width=\textwidth]{evoTimelines/evoTimelineCPP-3300/CPP-W3300-S3300_}
\end{figure*}

\begin{figure*}
   \centering
        \includegraphics[width=\textwidth]{evoTimelines/evoTimelineCPP-9900/CPP-W9900-S9900}
\end{figure*}
\begin{figure*}
   \centering
   \includegraphics[width=\textwidth]{evoTimelines/evoTimelineCPP-9900/CPP-W9900-S9900_}
\end{figure*}



\FloatBarrier
\subsection{LAD list}

\begin{figure*}
   \centering
        \includegraphics[width=\textwidth]{evoTimelines/evoTimelineLAD-50/LAD-W50-S200}
\end{figure*}
\begin{figure*}
   \centering
        \includegraphics[width=\textwidth]{evoTimelines/evoTimelineLAD-50/LAD-W50-S200_}
\end{figure*}

\begin{figure*}
   \centering
        \includegraphics[width=\textwidth]{evoTimelines/evoTimelineLAD-100/LAD-W100-S200}
\end{figure*}
\begin{figure*}
   \centering
        \includegraphics[width=\textwidth]{evoTimelines/evoTimelineLAD-100/LAD-W100-S200_}
\end{figure*}

\begin{figure*}
   \centering
        \includegraphics[width=\textwidth]{evoTimelines/evoTimelineLAD-250/LAD-W250-S250}
\end{figure*}
\begin{figure*}
   \centering
        \includegraphics[width=\textwidth]{evoTimelines/evoTimelineLAD-250/LAD-W250-S250_}
\end{figure*}

\begin{figure*}
   \centering
        \includegraphics[width=\textwidth]{evoTimelines/evoTimelineLAD-500/LAD-W500-S500}
\end{figure*}
\begin{figure*}
   \centering
        \includegraphics[width=\textwidth]{evoTimelines/evoTimelineLAD-500/LAD-W500-S500_}
\end{figure*}

\begin{figure*}
   \centering
        \includegraphics[width=\textwidth]{evoTimelines/evoTimelineLAD-1000/LAD-W1000-S1000}
\end{figure*}
\begin{figure*}
   \centering
        \includegraphics[width=\textwidth]{evoTimelines/evoTimelineLAD-1000/LAD-W1000-S1000_}
\end{figure*}

\begin{figure*}
   \centering
        \includegraphics[width=\textwidth]{evoTimelines/evoTimelineLAD-3300/LAD-W3300-S3300}
\end{figure*}
\begin{figure*}
   \centering
        \includegraphics[width=\textwidth]{evoTimelines/evoTimelineLAD-3300/LAD-W3300-S3300_}
\end{figure*}

\begin{figure*}
   \centering
        \includegraphics[width=\textwidth]{evoTimelines/evoTimelineLAD-9900/LAD-W9900-S9900}
\end{figure*}
\begin{figure*}
   \centering
        \includegraphics[width=\textwidth]{evoTimelines/evoTimelineLAD-9900/LAD-W9900-S9900_}
\end{figure*}


\FloatBarrier
\section{Stability in networks from Twitter, Facebook, Participabr}\label{si:ext}
To ease hypothesizing about the generality of the reported stability of human interaction networks,
this section presents the topological analysis of
networks from Twitter, Facebook and Participabr.
Selected networks are summarized in
Table~\ref{tab:E}. Their Erd\"os sector relative sizes are given in Table~\ref{tab:secE}. The formation of Principal components
are given in
Tables~\ref{tab:pcaE1F},~\ref{tab:pcaE1I},~\ref{tab:pcaE2} and~\ref{tab:pcaE3}. The friendship networks considered are undirected and unweighted, therefore all measurements of strength, in- and out- centralities, asymmetry and disequilibrium have little or no meaning, which is why F1, F2, F3, F4 and F5 are only present in Table~\ref{tab:pcaE1F}.
The most important results from this analysis are:

\begin{itemize}
%	\item the stability reported for email interaction networks is also valid for a broader class of phenomena.
	\item a further indicative that the stability reported with a focus on email interaction networks is valid for a broader class of phenomena.
	\item The stability in email interaction networks is higher than for the other networks, considering the same number of participants. This is especially important for benchmarking and probing general properties.
\end{itemize}

\begin{table*}[!h]
	\caption{Selected networks from three social platforms: Facebook, Twitter and Participabr.
	Both friendship and interaction networks were observed, yielding undirected and directed networks, respectively.
The number of agents $N$ and the number of edges $z$ are given on the last columns.
The acronyms, one for each network, are used throughout Tables~\ref{tab:secE},~\ref{tab:pcaE1I},~\ref{tab:pcaE1F},~\ref{tab:pcaE2} and~\ref{tab:pcaE3}. All the data were collected in 2013 and 2014 within the anthropological physics framework~\cite{anPhy}.}
\begin{center}
\begin{tabular}{| l || c | c | c | c | c | c | }\hline
	acronym & provenance & edge & directed & description & $N$ & $z$ \\ \hline
	F1 & Facebook  & friendship  & no  & the friendship network of Renato Fabbri (author)  & 1367  & 28606 \\\hline
F2 & Facebook  & friendship  & no  & the friendship network of Massimo Canevacci (senior anthropologist)  & 4764  & 59995 \\\hline
F3 & Facebook  & friendship  & no  & the friendship network of a brazilian direct democracy group  & 3599  & 59471 \\\hline
F4 & Facebook  & friendship  & no  & the friendship network of the Silicon Valley Global Network group  & 2026  & 15586 \\\hline
F5 & Participa.br  & friendship  & no  & the friendship network of a brazilian federal social participation portal  & 443  & 910 \\\hline
I1 & Facebook  & interaction  & yes  & the interaction network of the Silicon Valley Global Network group  & 104  & 154 \\\hline
I2 & Facebook  & interaction  & yes  & the interaction network of a Solidarity Economy group  & 64  & 120 \\\hline
I3 & Facebook  & interaction  & yes  & the interaction network of a brazilian direct democracy group  & 214  & 310 \\\hline
I4 & Facebook  & interaction  & yes  & the interaction network of the 'Cience with Frontiers' group  & 530  & 1658 \\\hline
I5 & Participa.br  & interaction  & yes  & the interaction network of a brazilian federal social participation portal  & 222  & 300 \\\hline
TT1 & Twitter  & retweet  & yes  & the retweet network of $\approx 22k$ tweets with the hashtag \#arenaNETmundial  & 2772  & 7222 \\\hline
TT2 & Twitter  & retweet  & yes  & same as TT1, but disconnected agents are not discarded  & 2975  & 7222 \\\hline

\hline
\end{tabular}
\end{center}
\label{tab:E}
\end{table*}


\begin{table*}[!h]
	\caption{
	Percentage of agents in each Erd\"os sector in the friendship and interaction human networks of Table~\ref{tab:E}. The ratios found in email networks are preserved. I1 and I4 are outliers, probably because they should be better characterized as a superposition of networks, rather than one coherent network. The degree was used for establishing the sectors.
	}
\begin{center}
\begin{tabular}{| l || c | c | c |}\hline
	 & periphery & intermediary & hubs \\ \hline
	F1 & 53.11  & 43.31  & 3.58 \\\hline
F2 & 58.98  & 39.29  & 1.72 \\\hline
F3 & 65.41  & 31.87  & 2.72 \\\hline
F4 & 66.49  & 32.03  & 1.48 \\\hline
F5 & 62.98  & 36.12  & 0.90 \\\hline
I1 & 4.81  & 94.23  & 0.96 \\\hline
I2 & 53.12  & 45.31  & 1.56 \\\hline
I3 & 58.41  & 40.19  & 1.40 \\\hline
I4 & 39.06  & 59.43  & 1.51 \\\hline
I5 & 54.95  & 43.69  & 1.35 \\\hline
TT1 & 74.86  & 24.49  & 0.65 \\\hline
TT2 & 76.57  & 22.86  & 0.57 \\\hline

\hline
\end{tabular}
\end{center}
\label{tab:secE}
\end{table*}


\begin{table*}[!h]
	\caption{Formation of first three principal components for each of the five friendship networks of Table~\ref{tab:E} in the simplest case: dimensions correspond to degree $k$, clustering coefficient $cc$ and betweenness centrality $bt$. Participabr yields the networks that most resemble the email networks.
		Overall, the general characteristic is preserved: first component is an average of degree and betweenness, while clustering is the most relevant for the second component. The friendship network of Renato Fabbri (F1) is the only network whose first component has more than 20\% of clustering coefficient and second component has more than 40\% of degree centrality.}
	\footnotesize
\begin{center}
%\begin{tabular}{| l | c |c |c |c |c |c |c |c |c |c |c |c |c |c |c |c |c |c |c |c |c |c |c |c |c |c |c |c |c |c |c |c |c | c | c | c |}\hline
\begin{tabular}{| l ||  c |c |c |c |c || c | c | c | c | c || c |c |c |c |c |	}\cline{2-16}
%	 & p. & i. & h. \\ \hline\hline
\multicolumn{1}{c|}{} & \multicolumn{5}{c||}{PC1}          & \multicolumn{5}{c||}{PC2} & \multicolumn{5}{c|}{PC3}  \\\cline{2-16}
	\multicolumn{1}{c|}{} & 
	F1 & F2 & F3 & F4 & F5 &	
	F1 & F2 & F3 & F4 & F5 &	
	F1 & F2 & F3 & F4 & F5 	\\\hline
	$cc$ & 25.80  & 12.22  & 11.54  & 5.04  & 0.94  & 58.87  & 78.22  & 68.86  & 90.39  & 88.86  & 6.95  & 1.13  & 18.20  & 3.02  & 5.90 \\
$k$ & 36.43  & 43.96  & 45.61  & 47.40  & 49.50  & 25.66  & 9.98  & 6.10  & 7.63  & 6.42  & 44.94  & 49.52  & 42.00  & 48.41  & 47.02 \\
$bt$ & 37.77  & 43.82  & 42.85  & 47.56  & 49.56  & 15.46  & 11.80  & 25.04  & 1.98  & 4.72  & 48.11  & 49.35  & 39.80  & 48.57  & 47.08 \\\hline\hline
$\lambda$ & 53.15  & 53.06  & 46.26  & 55.36  & 63.80  & 28.69  & 32.57  & 34.27  & 33.25  & 33.57  & 18.16  & 14.37  & 19.47  & 11.38  & 2.63 \\

\hline
\end{tabular}
\end{center}
\label{tab:pcaE1F}
\end{table*}

\begin{table*}[!h]
	\caption{Formation of the first three principal components for each of the seven interaction networks of Table~\ref{tab:E} in the simplest case: dimensions correspond to degree $k$, clustering coefficient $cc$ and betweenness centrality $bt$. Twitter yields the networks that most resemble the email networks. Overall, the general characteristic is preserved: first component is an average of degree and betweenness, while clustering is the most relevant for the second component.}
	\footnotesize
	\begin{center}
		%\begin{tabular}{| l | c |c |c |c |c |c |c |c |c |c |c |c |c |c |c |c |c |c |c |c |c |c |c |c |c |c |c |c |c |c |c |c |c | c | c | c |}\hline
		\begin{tabular}{| l ||  c |c |c |c |c | c | c || c | c | c | c | c | c | c || c |c |c |c |c | c | c |	}\cline{2-22}
			%	 & p. & i. & h. \\ \hline\hline
			\multicolumn{1}{c|}{} & \multicolumn{7}{c||}{PC1}          & \multicolumn{7}{c||}{PC2} & \multicolumn{7}{c|}{PC3}  \\\cline{2-22}
			\multicolumn{1}{c|}{} & 
			I1 & I2 & I3 & I4 & I5 & TT1 & TT2 &
			I1 & I2 & I3 & I4 & I5 & TT1 & TT2 &
			I1 & I2 & I3 & I4 & I5 & TT1 & TT2 \\\hline
			$cc$ & 14.43  & 17.12  & 11.54  & 0.69  & 13.26  & 2.17  & 2.72  & 74.78  & 70.72  & 79.30  & 96.63  & 76.59  & 95.75  & 94.69  & 1.58  & 4.09  & 2.46  & 1.71  & 0.57  & 2.03  & 2.20 \\
$k$ & 42.68  & 41.77  & 44.37  & 49.65  & 43.41  & 48.94  & 48.67  & 13.85  & 11.48  & 8.31  & 2.35  & 11.26  & 0.14  & 0.52  & 49.07  & 48.42  & 48.94  & 49.14  & 49.76  & 49.01  & 48.93 \\
$bt$ & 42.89  & 41.11  & 44.09  & 49.66  & 43.34  & 48.89  & 48.61  & 11.37  & 17.80  & 12.39  & 1.02  & 12.15  & 4.12  & 4.79  & 49.35  & 47.49  & 48.60  & 49.15  & 49.67  & 48.96  & 48.87 \\\hline\hline
$\lambda$ & 64.58  & 61.97  & 56.95  & 62.01  & 50.92  & 64.82  & 64.83  & 31.57  & 30.98  & 32.56  & 33.35  & 32.51  & 33.33  & 33.32  & 3.85  & 7.05  & 10.50  & 4.64  & 16.57  & 1.85  & 1.86 \\

			\hline
		\end{tabular}
	\end{center}
	\label{tab:pcaE1I}
\end{table*}


\begin{table*}[!h]
	\caption{Formation of the first three principal components for each of the seven interaction networks of Table~\ref{tab:E} considering total, in- and out- degrees ($k$, $k^{in}$, $k^{out}$) and strengths ($s$, $s^{in}$, $s^{out}$), clustering coefficient $cc$ and betweenness centrality $bt$.
	Twitter yields the networks that most resemble email networks.
The general characteristic is preserved: first component is an average of degree and betweenness, while clustering coefficient is the most relevant for the second component.
Important differences are:
the clustering coefficient was only important to the third component for two of the networks ($I2$, $I3$) and does not contribute significantly to any of the first three principal components in $I5$;
in the first component, I5 exhibited less contribution from in-strength, in-degree and betweenness, I4 exhibited less contribution from out-degree.}
	\footnotesize
	\begin{center}
		%\begin{tabular}{| l | c |c |c |c |c |c |c |c |c |c |c |c |c |c |c |c |c |c |c |c |c |c |c |c |c |c |c |c |c |c |c |c |c | c | c | c |}\hline
		\begin{tabular}{| l ||  c |c |c |c |c | c | c || c | c | c | c | c | c | c || c |c |c |c |c | c | c |	}\cline{2-22}
			%	 & p. & i. & h. \\ \hline\hline
			\multicolumn{1}{c|}{} & \multicolumn{7}{c||}{PC1}          & \multicolumn{7}{c||}{PC2} & \multicolumn{7}{c|}{PC3}  \\\cline{2-22}
			\multicolumn{1}{c|}{} & 
			I1 & I2 & I3 & I4 & I5 & TT1 & TT2 &
			I1 & I2 & I3 & I4 & I5 & TT1 & TT2 &
			I1 & I2 & I3 & I4 & I5 & TT1 & TT2 \\\hline
			$cc$ & 2.79  & 4.34  & 2.57  & 0.82  & 1.29  & 0.66  & 0.76  & 28.44  & 9.46  & 3.29  & 21.95  & 6.95  & 29.82  & 30.04  & 32.24  & 60.89  & 80.24  & 43.85  & 3.81  & 33.84  & 33.54 \\\hline
$s$ & 15.28  & 15.84  & 16.46  & 16.01  & 16.70  & 15.49  & 15.47  & 3.78  & 4.95  & 2.90  & 3.26  & 17.78  & 1.95  & 2.05  & 1.95  & 0.34  & 0.87  & 4.84  & 11.15  & 0.52  & 0.43 \\
$s^{in}$ & 14.48  & 12.81  & 13.62  & 14.63  & 4.50  & 11.85  & 11.84  & 11.77  & 18.29  & 17.41  & 12.44  & 16.19  & 19.03  & 18.81  & 5.38  & 5.03  & 0.93  & 11.16  & 30.41  & 21.48  & 21.71 \\
$s^{out}$ & 12.13  & 12.12  & 12.59  & 12.91  & 19.02  & 13.87  & 13.85  & 17.19  & 16.79  & 20.12  & 18.81  & 8.90  & 13.42  & 13.43  & 19.35  & 7.90  & 3.11  & 11.38  & 14.58  & 12.91  & 12.92 \\
$k$ & 15.32  & 16.22  & 16.12  & 16.20  & 21.12  & 15.48  & 15.46  & 3.13  & 4.18  & 6.25  & 2.88  & 9.26  & 3.32  & 3.24  & 1.84  & 0.09  & 1.16  & 0.11  & 2.22  & 4.26  & 4.30 \\
$k^{in}$ & 14.49  & 13.56  & 12.90  & 15.34  & 7.29  & 12.99  & 12.98  & 10.45  & 16.50  & 19.68  & 11.13  & 20.75  & 17.89  & 17.86  & 8.78  & 4.07  & 1.26  & 6.07  & 15.41  & 14.67  & 14.65 \\
$k^{out}$ & 11.70  & 11.25  & 11.80  & 9.24  & 21.09  & 14.20  & 14.19  & 19.14  & 20.50  & 21.19  & 26.13  & 0.19  & 12.36  & 12.28  & 18.80  & 7.50  & 4.68  & 20.44  & 10.57  & 12.14  & 12.20 \\
$bt$ & 13.82  & 13.86  & 13.93  & 14.86  & 8.99  & 15.47  & 15.45  & 6.10  & 9.32  & 9.16  & 3.41  & 19.99  & 2.20  & 2.29  & 11.66  & 14.20  & 7.75  & 2.17  & 11.86  & 0.18  & 0.25 \\\hline\hline
$\lambda$ & 71.73  & 60.58  & 60.35  & 64.53  & 41.28  & 70.06  & 70.08  & 15.23  & 21.53  & 20.13  & 16.42  & 22.83  & 13.83  & 13.86  & 9.95  & 11.37  & 12.25  & 11.19  & 15.71  & 11.43  & 11.38 \\

			\hline
		\end{tabular}
	\end{center}
	\label{tab:pcaE2}
\end{table*}


\begin{table*}[!h]
	\caption{Formation of the first three principal components for each of the seven interaction networks of Table~\ref{tab:E} considering total, in- and out- degrees ($k$, $k^{in}$, $k^{out}$) and strengths ($s$, $s^{in}$, $s^{out}$), clustering coefficient $cc$, betweenness centrality $bt$ and symmetry related metrics ($asy$, $\mu^{asy}$, $\sigma^{asy}$, $dis$, $\mu^{dis}$ and $\sigma^{dis}$ defined in Section~\ref{measures}).
	The characteristics found in email interaction networks are preserved: the first component is an average of degree and betweenness, the second component is mostly governed by symmetry related metrics, and clustering coefficient is mostly relevant for the third component.
Standard deviation of asymmetry and disequilibrium metrics are again coupled to clustering coefficient in the third component.
Important differences are: 
the first component is a less regular average of centrality measures and has a greater contribution of symmetry metrics;
The first component of I5 is formed mostly from symmetry, not centrality, metrics.}
	\footnotesize
	\begin{center}
		%\begin{tabular}{| l | c |c |c |c |c |c |c |c |c |c |c |c |c |c |c |c |c |c |c |c |c |c |c |c |c |c |c |c |c |c |c |c |c | c | c | c |}\hline
		\begin{tabular}{| l ||  c |c |c |c |c | c | c || c | c | c | c | c | c | c || c |c |c |c |c | c | c |	}\cline{2-22}
			%	 & p. & i. & h. \\ \hline\hline
			\multicolumn{1}{c|}{} & \multicolumn{7}{c||}{PC1}          & \multicolumn{7}{c||}{PC2} & \multicolumn{7}{c|}{PC3}  \\\cline{2-22}
			\multicolumn{1}{c|}{} & 
			I1 & I2 & I3 & I4 & I5 & TT1 & TT2 &
			I1 & I2 & I3 & I4 & I5 & TT1 & TT2 &
			I1 & I2 & I3 & I4 & I5 & TT1 & TT2 \\\hline
			$cc$ & 3.46  & 4.19  & 2.44  & 0.36  & 2.18  & 1.28  & 1.17  & 3.06  & 1.61  & 1.23  & 1.19  & 2.57  & 3.03  & 2.17  & 17.36  & 16.88  & 21.68  & 17.00  & 10.00  & 18.65  & 19.13 \\\hline
$s$ & 10.05  & 9.21  & 9.60  & 9.31  & 3.54  & 10.27  & 10.59  & 5.81  & 5.74  & 7.33  & 8.47  & 9.24  & 6.26  & 5.96  & 4.58  & 8.02  & 4.91  & 2.21  & 13.10  & 0.92  & 1.53 \\
$s^{in}$ & 9.57  & 8.03  & 9.21  & 8.74  & 0.78  & 7.75  & 7.99  & 4.63  & 0.59  & 1.27  & 6.69  & 2.77  & 5.38  & 5.29  & 8.22  & 12.82  & 9.18  & 6.53  & 7.63  & 5.90  & 4.77 \\
$s^{out}$ & 7.88  & 6.21  & 5.45  & 6.97  & 5.76  & 9.25  & 9.54  & 6.78  & 10.23  & 12.76  & 9.20  & 10.26  & 5.27  & 4.92  & 5.90  & 2.99  & 3.86  & 8.43  & 10.84  & 4.58  & 4.76 \\
$k$ & 10.44  & 10.02  & 9.88  & 10.39  & 5.80  & 10.80  & 11.05  & 4.62  & 5.13  & 5.66  & 5.54  & 14.08  & 4.48  & 4.29  & 3.63  & 6.02  & 6.18  & 1.86  & 1.21  & 1.33  & 1.30 \\
$k^{in}$ & 10.12  & 9.30  & 9.50  & 9.98  & 4.43  & 8.64  & 8.86  & 2.69  & 0.70  & 0.88  & 4.49  & 9.61  & 5.40  & 5.50  & 7.12  & 10.55  & 8.70  & 6.17  & 8.24  & 7.27  & 6.54 \\
$k^{out}$ & 7.27  & 5.29  & 4.43  & 5.43  & 9.11  & 10.10  & 10.33  & 8.36  & 12.52  & 13.63  & 5.65  & 11.61  & 3.38  & 3.08  & 7.82  & 5.77  & 2.07  & 13.52  & 5.68  & 5.00  & 4.65 \\
$bt$ & 9.62  & 7.97  & 7.53  & 8.93  & 2.25  & 10.47  & 10.78  & 3.77  & 8.42  & 9.14  & 6.95  & 8.12  & 5.60  & 5.29  & 2.72  & 0.42  & 1.99  & 2.74  & 8.66  & 1.16  & 1.60 \\\hline
$asy$ & 5.42  & 7.05  & 7.97  & 8.48  & 15.47  & 6.16  & 5.79  & 14.17  & 12.88  & 11.78  & 11.02  & 4.67  & 12.48  & 13.39  & 2.95  & 1.03  & 0.58  & 2.71  & 0.87  & 6.54  & 5.80 \\
$\mu^{asy}$ & 5.48  & 6.99  & 7.99  & 8.47  & 15.44  & 6.18  & 5.80  & 14.12  & 13.04  & 11.78  & 11.01  & 4.72  & 12.46  & 13.37  & 2.92  & 0.76  & 0.75  & 2.77  & 0.76  & 6.58  & 5.83 \\
$\sigma^{asy}$ & 6.53  & 7.39  & 7.63  & 7.15  & 2.37  & 5.59  & 5.48  & 1.69  & 3.80  & 1.75  & 8.46  & 7.49  & 5.94  & 5.45  & 11.32  & 8.91  & 11.14  & 3.04  & 15.54  & 13.70  & 15.31 \\
$dis$ & 5.02  & 6.67  & 7.78  & 8.08  & 15.41  & 5.98  & 5.59  & 14.12  & 13.41  & 11.92  & 11.53  & 4.80  & 12.45  & 13.38  & 4.99  & 1.40  & 0.67  & 3.02  & 0.83  & 7.44  & 6.69 \\
$\mu^{dis}$ & 5.33  & 7.01  & 7.24  & 6.92  & 14.34  & 5.49  & 5.14  & 13.33  & 10.15  & 9.47  & 8.02  & 5.05  & 11.86  & 12.65  & 1.66  & 7.08  & 5.72  & 11.38  & 2.68  & 0.77  & 0.66 \\
$\sigma^{dis}$ & 3.82  & 4.68  & 3.34  & 0.81  & 3.12  & 2.03  & 1.88  & 2.85  & 1.77  & 1.39  & 1.77  & 5.00  & 6.01  & 5.24  & 18.82  & 17.36  & 22.58  & 18.61  & 13.97  & 20.16  & 21.42 \\\hline\hline
$\lambda$ & 46.11  & 43.48  & 44.29  & 46.95  & 30.34  & 44.12  & 43.52  & 26.42  & 24.97  & 24.76  & 19.99  & 23.91  & 25.98  & 26.13  & 14.90  & 14.72  & 11.82  & 13.16  & 17.32  & 11.62  & 12.15 \\

			\hline
		\end{tabular}
	\end{center}
	\label{tab:pcaE3}
\end{table*}




%\newpage
%\pagebreak
\clearpage
%\nocite{*}
\bibliography{supportingInformation}% Produces the bibliography via BibTeX.
\end{document}
%
% ****** End of file aipsamp.tex ******


